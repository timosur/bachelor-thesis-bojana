%-----------------------------------
% Define document and include general packages
%-----------------------------------
\documentclass[12pt,oneside,titlepage,listof=totoc,bibliography=totoc]{scrartcl}
\usepackage[utf8]{inputenc}
\usepackage[ngerman]{babel}
\usepackage[babel,german=quotes]{csquotes}
\usepackage[T1]{fontenc}
\usepackage{fancyhdr}
\usepackage{fancybox}
\usepackage[a4paper, left=3cm, right=3cm, top=3cm, bottom=3cm]{geometry}
\usepackage{graphicx}
\usepackage{appendix}
\usepackage{colortbl}
\usepackage{array}
\usepackage{float}      %Positionierung von Abb. und Tabellen mit [H] erzwingen
\usepackage{footnote}
\usepackage{caption}
\usepackage{mdwlist}
\usepackage{amssymb}
\usepackage{mathptmx}
\usepackage{amsmath}
\usepackage[table]{xcolor}
\usepackage{marvosym}			% Verwendung von Symbolen, z.B. perfektes Eurozeichen
\usepackage[colorlinks=true,linkcolor=black]{hyperref}
\definecolor{darkblack}{rgb}{0,0,0}
\hypersetup{colorlinks=true, breaklinks=true, linkcolor=darkblack, menucolor=darkblack, urlcolor=darkblack}
\usepackage{times}
\usepackage{enumitem}
\fontfamily{ptm}\selectfont

	\usepackage{fontspec}
\setmainfont{Times New Roman}

% Mehrere Fussnoten nacheinander mit Komma separiert
\usepackage[multiple]{footmisc}
\usepackage{todonotes}

%Pakete für Tabellen
\usepackage{epstopdf}
\usepackage{nicefrac} % Brüche
\usepackage{multirow}
\usepackage{rotating} % vertikal schreiben
\usepackage{colortbl}
\usepackage{mdwlist}

\definecolor{dunkelgrau}{rgb}{0.8,0.8,0.8}
\definecolor{hellgrau}{rgb}{0.0,0.7,0.99}
% Colors for listings
\definecolor{mauve}{rgb}{0.58,0,0}
\definecolor{dkgreen}{rgb}{0,0.6,0}

% sauber formatierter Quelltext
\usepackage{listings}
\lstset{numbers=left,
	numberstyle=\tiny,
	numbersep=5pt,
	breaklines=false,
	showstringspaces=false,
	xleftmargin=10pt,
	xrightmargin=5pt,
	basicstyle=\ttfamily\scriptsize,
	stepnumber=1,
	keywordstyle=\color{blue},          % keyword style
  	commentstyle=\color{dkgreen},       % comment style
  	stringstyle=\color{mauve}         % string literal style
}

\lstdefinestyle{nonumbers}
{numbers=none}

% Biblatex
\usepackage[
backend=biber,
style=numeric,
citestyle=authoryear,
url=false,
isbn=false,
notetype=footonly,
hyperref=false,
sortlocale=de]{biblatex}

%weitere Anpassungen für BibLaTex
% Opptionen für Biblatex
\ExecuteBibliographyOptions{%
giveninits=false,
isbn=true, 
url=true, 
doi=false, 
eprint=false,
maxbibnames=7, % Alle Autoren (kein et al.)
maxcitenames=2, % et al. ab dem 3. Autor
backref=false, % Rückverweise auf Zitatseiten
bibencoding=utf8, % wenn .bib in utf8, sonst ascii
bibwarn=true, % Warnung bei fehlerhafter bib-Datei
}%

% et al. an Stelle von u.a.
\DefineBibliographyStrings{ngerman}{ 
   andothers = {{et\,al\adddot}},             
}

% Klammern um das Jahr in der Fußnote
\renewbibmacro*{cite:labelyear+extrayear}{% 
  \iffieldundef{labelyear} 
    {} 
    {\printtext[bibhyperref]{% 
       \mkbibparens{% 
         \printfield{labelyear}% 
         \printfield{extrayear}}}}}

\DeclareNameFormat{last-first}{%
  \iffirstinits
    {\usebibmacro{name:family-given}
        {\namepartfamily}
        {\namepartgiveni}
        {\namepartprefix}
        {\namepartsuffix}
    }
    {\usebibmacro{name:family-given}
        {\namepartfamily}
        {\namepartgiven}
        {\namepartprefix}
        {\namepartsuffix}
    }%
  \usebibmacro{name:andothers}}

% Alternative Notation der Fußnoten 
% Zeigt sowohl den Nachnamen als auch den Vornamen an
% Beispiel: \fullfootcite[Vgl. ][Seite 5]{Tanenbaum.2003} 
\DeclareCiteCommand{\fullfootcite}[\mkbibfootnote]
  {\usebibmacro{prenote}}
  {\usebibmacro{citeindex}%
    \printnames[sortname][1-1]{author}%
    \addspace (\printfield{year})}
  {\addsemicolon\space}
  {\usebibmacro{postnote}}

%Autoren (Nachname, Vorname)
\DeclareNameAlias{default}{family-given}

%Reihenfolge von publisher, year, address verändern
% Achtung, bisher nur für den Typ @book definiert

%% Definiert @Book Eintrag
\DeclareBibliographyDriver{book}{%
  \printnames{author}%
  \newunit\addcolon\space
  \printfield{title}%
  \setunit*{,\space}%
  \printfield{edition}%
  \setunit*{\addcomma\space}%
  \printlist{publisher}%
  \newunit\newblockpunct
  \printlist{location}%
  \setunit*{\space}%
  \printfield{year}%
  \setunit*{,\space}% 
  \printfield{isbn}%
  \finentry}

%% Definiert @Online Eintrag
\DeclareBibliographyDriver{online}{%
  \printnames{author}%
  \newunit\newblockpunct
  \printfield{title}%
  \setunit*{,\space}%
  %\newunit\newblock
  \printfield{url}%
  \setunit*{,\space Erscheinungsjahr:\space}%
  \printfield{year}%
  \setunit*{,\space Aufruf am:\space}%
  \printfield{note}%
  \finentry}
  
%% Definiert @Article Eintrag
\DeclareBibliographyDriver{article}{%
  \printnames{author}%
  \newunit\newblockpunct
  \printfield{title}%
  \setunit*{.\space In:\space}%
  %\newunit\newblock
  \usebibmacro{journal}%
  \setunit*{\space (}%
  \printfield{year}\newunit{)}%
  \finentry}  

%Doppelpunkt nach dem letzten Autor
\renewcommand*{\labelnamepunct}{\addcolon\addspace }

%Komma an Stelle des Punktes
\renewcommand*{\newunitpunct}{\addcomma\space}

%Autoren durch Semikolon trennen
\newcommand*{\bibmultinamedelim}{\addsemicolon\space}% 
\newcommand*{\bibfinalnamedelim}{\addsemicolon\space}% 
\AtBeginBibliography{% 
  \let\multinamedelim\bibmultinamedelim 
  \let\finalnamedelim\bibfinalnamedelim 
}

%Titel nicht kursiv anzeigen 
\DeclareFieldFormat{title}{#1\isdot}


%Bib-Datei einbinden
\addbibresource{./literatur/literatur.bib}

% Pfad fuer Abbildungen
\graphicspath{{./}{./abbildungen/}}

%-----------------------------------
% Weitere Ebene einfügen
\usepackage{titletoc}

\makeatletter

% Setze die Tiefe des Inhaltsverzeichnis auf 4 Ebenen
% Damit erscheinen \paragraph-Sektionen auch im Inhaltsverzeichnis
\setcounter{secnumdepth}{4}
\setcounter{tocdepth}{4}

% Fuege Abstand nach unten wie in einer normalen \section hinzu
% Andernfalls haette \paragraph keinen Zeilenumbruch
% Der Zeilenumbruch koennte mit einer leeren \mbox{} ersetzt werden
% Jedoch klebt dann der Text relativ nah an der Ueberschrift
\renewcommand{\paragraph}{%
  \@startsection{paragraph}{4}%
  {\z@}{3.25ex \@plus 1ex \@minus .2ex}{1.5ex plus 0.2ex}%
  {\normalfont\normalsize\bfseries\sffamily}%
}

\makeatother


%-----------------------------------
% Zeilenabstand 1,5-zeilig
%-----------------------------------
\usepackage{setspace}
\onehalfspacing

%-----------------------------------
% Absätze durch eine neue Zeile
%-----------------------------------
\setlength{\parindent}{0mm}
\setlength{\parskip}{0.8em plus 0.5em minus 0.3em}

\sloppy					%Abstände variieren
\pagestyle{headings}

%-----------------------------------
% Abkürzungsverzeichnis
%-----------------------------------
\usepackage[intoc]{nomencl}
\renewcommand{\nomname}{Abkürzungsverzeichnis}
\setlength{\nomlabelwidth}{.20\textwidth}
\renewcommand{\nomlabel}[1]{#1 \dotfill}
\setlength{\nomitemsep}{-\parsep}
\makenomenclature

%-----------------------------------
% Meta informationen
%-----------------------------------
%-----------------------------------
% Meta Informationen zur Arbeit
%-----------------------------------

% Autor
\newcommand{\myAutor}{Bojana Kapetanovic}

% Adresse
\newcommand{\myAdresse}{Daimlerstraße 18 \\ \> \> 45661 Recklinghausen}

% Titel der Arbeit
\newcommand{\myTitel}{„Prüfungsstile: zwischen Kooperation und Konfrontation. Handlungsspielräume in mündlichen Prüfungen kennen“ aus „Der Prüfer ist nicht der König: Mündliche Abschlussprüfungen in der Hochschule“
von Prof. Dr. Dorothee Meer}

% Dozent
\newcommand{\myBetreuer}{Tim Heizen}

% Lehrveranstaltung
\newcommand{\myLehrveranstaltung}{Mediative Kommunikation}

% Matrikelnummer
\newcommand{\myMatrikelNr}{5000329}

% Fachsemester
\newcommand{\myFachsemester}{6. Fachsemester}

% Ort
\newcommand{\myOrt}{Bochum}

% Datum der Abgabe
\newcommand{\myAbgabeDatum}{24. Juli 2020}

% Semesterzahl
\newcommand{\mySemesterZahl}{6}

% Name der Hochschule
\newcommand{\myHochschulName}{Diploma Hochschule}

% Standort der Hochschule
\newcommand{\myHochschulStandort}{Standort Bochum}

% Studiengang
\newcommand{\myStudiengang}{Grafik Design}

% Art der Arbeit
\newcommand{\myThesisArt}{Schriftliche Ausarbeitung}

% Zu erlangender akademische Grad
\newcommand{\myAkademischerGrad}{Bachelor of Science (B. Sc.)}

% Firma
\newcommand{\myFirma}{KPS AG}

%-----------------------------------
% Footer definieren
%-----------------------------------
\pagestyle{fancy}
\fancyhf{}
\renewcommand{\headrulewidth}{0pt}
\fancyfoot{}
\renewcommand{\footrulewidth}{0pt}
\fancyfoot[R]{\thepage}
  
%-----------------------------------
% Start the document here:
%-----------------------------------
\begin{document}

\pagenumbering{Roman}								% Seitennumerierung auf römisch umstellen
\newcolumntype{C}{>{\centering\arraybackslash}X}		% Neuer Tabellen-Spalten-Typ:
													% Zentriert und umbrechbar

%-----------------------------------
% Titlepage
%-----------------------------------
\begin{titlepage}
	%\newgeometry{left=2cm, right=2cm, top=2cm, bottom=2cm}
	\begin{center}
		\textbf{\myHochschulName}\\
		\textbf{\myHochschulStandort}\\
%		\vspace{1.5cm}
%			\includegraphics[width=3cm]{abbildungen/fomLogo.jpg} \\
		\vspace{2cm}
		Berufsbegleitender Studiengang\\
		\myStudiengang, \mySemesterZahl. Semester\\
		\vspace{1cm}
		\textbf{\myThesisArt}\\
%		\textbf{zur Erlangung des Grades eines}\\
%		\textbf{\myAkademischerGrad}\\
		% Oder für Hausarbeiten:
		\textbf{im Rahmen des Moduls}\\
		\textbf{\myLehrveranstaltung}\\
		\vspace{1cm}
		zu dem Text\\
		\large{\myTitel}\\
		\vspace{0.2cm}
	\end{center}
	\normalsize
	\vfill
	\begin{tabbing}
		Links \= Mitte \= Rechts\kill
		Betreuer: \> \> \myBetreuer\\
		\> \> \\

		Autor: \> \> \myAutor\\
		\> \>  Matrikelnr.: \myMatrikelNr\\
		\> \> \myFachsemester\\
		\> \> \myAdresse\\
		\> \> \\
		Abgabe: \> \> \myAbgabeDatum
	\end{tabbing}
\end{titlepage}

%-------Ende Titelseite-------------

%-----------------------------------
% Sperrvermerk
%-----------------------------------
%\newpage
\thispagestyle{empty}

%-----------------------------------
% Sperrvermerk
%-----------------------------------
\section*{Sperrvermerk}
Die vorliegende Abschlussarbeit mit dem Titel \enquote{\myTitel} enthält unternehmensinterne Daten der Firma \myFirma . Daher ist sie nur zur Vorlage bei der FOM sowie den Begutachtern der Arbeit bestimmt. Für die Öffentlichkeit und dritte Personen darf sie nicht zugänglich sein.

\par\medskip
\par\medskip

\_\_\_\_\_\_\_\_\_\_\_\_\_\_\_\_\_\_\_\_\_\_\_\_ \hspace{1.5cm} \_\_\_\_\_\_\_\_\_\_\_\_\_\_\_\_\_\_\_\_\_\_\_\_ \\
(Ort, Datum)\hspace{4.5cm}
(Eigenhändige Unterschrift)

\newpage

%-----------------------------------
% Inhaltsverzeichnis
%-----------------------------------
\setcounter{page}{2}
\tableofcontents\thispagestyle{fancy}
\newpage

%-----------------------------------
% Abbildungsverzeichnis
%-----------------------------------
%\listoffigures
%\newpage

%-----------------------------------
% Tabellenverzeichnis
%-----------------------------------
%\listoftables
%\newpage

%-----------------------------------
% Abkürzungsverzeichnis
%-----------------------------------
\printnomenclature
\newpage

%-----------------------------------
% Seitennummerierung auf arabisch und ab 1 beginnend umstellen
%-----------------------------------
\pagenumbering{arabic}
\setcounter{page}{1}

%-----------------------------------
% Kapitel / Inhalte
%-----------------------------------
\section{Einleitung}
Jährlich werden weltweit rund 1,3 Milliarden Tonnen Lebensmittel entlang der Wertschöpfungskette entsorgt. Dies entspricht ein Drittel aller weltweit produzierten Lebensmittel. $ (Quelle: FAO: http://www.fao.org/fileadmin/user_upload/newsroom/docs/FAO%20ruft%20dazu%20auf%20weniger%20Lebensmittel%20zu%20verschwenden.pdf) $
$
Allein in Deutschland fallen 18 Mio. t. Lebensmittelverluste jährlich an.(quelle WWF) Doch was ist die Ursache der Lebensmittelverschwendung in Deutschland? Dokumentarfilmer Valetin Thurn, bekannt für seinen Film „Taste the Waste“, äußerte sich in einem Interview folgendermaßen: „Es ist ein System, an dem wir alle unseren Anteil haben, in dem wir im Supermarkt eben nur Produkte kaufen, die besonders schön aussehen. Das führt zu kosmetischen Standards und die sorgen für Müll nicht nur im Supermarkt, sondern schon vorher in der Landwirtschaft, wo danach bereits aussortiert wird.“ $ (Quelle: http://www.planet-interview.de/interviews/valentin-thurn/35464/) $

Das Thema Lebensmittelverschwendung findet immer wieder mediale Beachtung, nicht nur durch Thurns Film. Auch andere Medien greifen das Thema auf: Presseberichte, Zeitungsbeiträge, Filme, Fernsehdokumentationen und Radio- und Podcastbeiträge.


Ziel der Arbeit ist es, die Ursachen der Lebensmittelverluste innerhalb Deutschlands auf allen Stufen aufzuzeigen. Des Weiteren sollen auch die Umwelteffekte die aus dem Verlust resultieren, dargelegt werden. Der Einfluss der Politik bzw. ihr Engagement gegen Lebensmittelverschwendung verdient in diesem Zusammenhang eine nähere Betrachtung.

Innerhalb des Praxisteils wird der Fokus aus die Lebensmittelverschwendung innerhalb Privathaushalten gelegt. Da in diesem Segment die vermeidbaren Verluste besonders hoch sind und der Konsument durch sein Kaufverhalten auch die anderen Bereiche der Wertschöpfungsketten beeinflusst. 
\newpage

\section{Definitionen}
\subsection{Lebensmittel}
Nach Art. 2 VO (EG) Nr. 178/2002 wird mit dem Begriff Lebensmittel dieses als Ganzes in allen Verarbeitungsgraden von der Erzeugung bis zum Verbrauch die sich für den menschlichen verzehr eignen beschrieben. Dazu zählen auch Getränke und jegliche Stoffe die dem Lebensmittel während seiner Herstellung bzw. Ver-/Bearbeitung hinzugefügt werden. 

\subsection{Lebensmittelverluste}
Als Lebensmittelverlust sind Lebensmittel oder Lebensmittelbestandteile, die die Lebensmittelkette verlassen, ohne das sie verzehrt werden können. Solche Verluste entstehen durch fehlerhafte Produktion, Lagerung oder Verpackung.  $(https://www.umwelt.nrw.de/fileadmin/redaktion/Broschueren/lebensmittelverschwendung_bf.pdf , s.8)$ 
\subsection{Lebensmittelabfälle}
Lebensmittelabfälle können in vermeidbare, teilweise vermeidbare und unvermeidbare Abfälle unterschieden werden. %TODO(Quelle: Fakt ist Lebensmittelverschwendung, S.4)
\begin{itemize}
 \item Vermeidbar sind die Abfälle, die bei ihrer Entsorgung noch genießbar wären z. B.: Original verpackte Lebensmittel. %TODO(Quelle: Fakt ist Lebensmittelverschwendung S.4) 
  \item Teilweise vermeidbare Abfälle bezeichnen Abfälle die durch Gewohnheiten der Verbraucher entstehen z. B. Speisereste, Brotrinde, Apfel oder Kartoffelschalen. %TODO (Quelle: Fakt ist Lebensmittelverschwendung S.4)
  \item Die Unvermeidbare Lebensmittelabfälle treten bei der Zubereitung auf. Sie bezeichnen nicht essbare teile der Lebensmittel z.B. Eierschalen, Knochen.%TODO (Koester, 2012, S. 1), (Kranert, et al., 2012, S. 13) $(Quelle: https://www.wwf.de/fileadmin/fm-wwf/Publikationen-PDF/WWF_Studie_Das_grosse_Wegschmeissen.pdf)$ 
 
\end{itemize}
 


\subsection{Wertschöpfungskette}
Die Lebensmittel Wertschöpfungskette wird in vier Schritte unterteilt:
\begin{itemize}
 \item Produktion und Erzeugung: Sobald Gemüse oder Obst geerntet, bei einer Kuh die Milch gemolken, ein Huhn ein Ei legt oder ein Tier geschlachtet bzw. Fisch gefangen wird, beginnt die Wertschöpfungskette. %TODO (Lipinski, et al., 2013, S. 4), (HLPE, 2014, S. 22)
 \item Weiterverarbeitung: Die technologische Verarbeitung von Lebensmittel beschreibt den nächsten Schritt.
  \item Finaler Konsum: In Privathaushalten oder im Bewirtungssektor endet die Wertschöpfungskette, wenn das Lebensmittel nicht auf anderem Wege aus der Kette entfernt wurde. %TODO (Lipinski, et al., 2013, S. 4), (HLPE, 2014, S. 22).

 
\end{itemize}

\subsection{Mindesthaltbarkeitsdatum}
Das Mindesthaltbarkeitsdatum ist die Zusicherung des Herstellers, dass ein Lebensmittel, bei ordnungsgemäßer Lagerung, seine spezifischen Eigenschaften (Geschmack, Geruch, Farbe, Textur etc.) bis zu dem gekennzeichneten Datum behält. Das Datum bezieht sich immer auf ungeöffnete Verpackungen. %TODO(Artikel 1 und 9 LMIV, VO (EU) Nr. 1169/2011), (BMEL, 2014) 


\subsection{Verbrauchsdatum}
Schnell verderbliche Lebensmittel erhalten ein Verbrauchsdatum, denn nach kurze Zeit können diese Lebensmittel( z. B. Hackfleisch oder Geflügel) Gesundheitsgefährdend sein. Ist das VBD überschritten, darf ein Lebensmittel nicht mehr verkauft oder verzehrt werden. (Artikel 24 LMIV, VO (EU) Nr. 1169/2011).
Die Vergabe eines MHD oder VBD ist für verpackte Produkte ist gesetzlich vorgeschrieben. Die Produzenten können aber dies selbst festlegen.%TODO (LMIV, VO (EU) 1169/2011


\section{Ursachen innerhalb der Wertschöpfungskette und Umwelteffekte von Lebensmittelverlusten und
-verschwendung}

\subsection{Ursachen innerhalb der Wertschöpfungskette}
\subsubsection{Landwirtschaft}
\paragraph{Ernteverluste}
Der Lebensmittel Verlust während der Ernte beläuft sich auf ca. 1 Mio. Tonnen, der vermeidbare Anteil ist relativ gering. %TODO (WWF das große wegschmeißen, s.10)

\subparagraph{Definition}	
Bereits bei der Erzeugung bzw. Nutzbarmachung der Lebensmittel entstehen Abfälle. Diese anfallenden Einbußen werden als Ernteverlust bezeichnet. Dazu gehören Verluste die durch mechanische Zerstörung oder Verschüttung zustande gekommen sind. Des Weiteren zählen nach der Ernte aussortierte pflanzliche Waren dazu. Dies geschieht aufgrund von Schäden oder Makeln in z. B. Farbe, Form, Größe. %TODO (Gustavsson et al., 2011).

Ernteverluste bei tierischen Lebensmittel beschreiben in erste Linie Verschütten(Milch) oder Zerbrechen(Eier) bei der Übergabe an den Abnehmer. %TODO $(Quelle: https://www.wwf.de/fileadmin/fm-wwf/Publikationen-PDF/WWF_Studie_Das_grosse_Wegschmeissen.pdf , S. 27)$  tote tiere?
 
\subparagraph{Überschussproduktion}	
Obwohl die meisten Handelsnormen offiziell abgeschafft wurden, verlangt der Handel immer noch nach normgerechter Ware bei pflanzlichen Lebensmitteln. Da der Landwirt nicht beeinflussen kann, wie viel seiner Ernte den Normen entspricht, produziert gezielt auf Überschuss. %TODO(Priefer & Jörissen, 2012, S. 32-33), (Gustavsson, et al., 2011, S. 11), (Göbel, et al., 2012, S. 14). 
Die überproduzierte Ware stellt aber insoweit ein Problem dar, das sie günstiger verkauft werden müsste, als die Normenkonforme Ware. Dies würde sich kritisch auf die Marktpreise auswirken. Zudem wäre es nicht rentabel, wenn der Preis niedriger als die Produktionskosten liegt. (Schneider F., 2008, S. 3) Ernteüberschüsse werden regulär für die Herstellung von Tierfuttermittel oder Kompost weiterverwendet.%TODO (Schneider F., 2008, S. 3) 
Auch wenn die überproduzierten Lebensmittel weiterverwendet werden, handele es sich trotzdem um Lebensmittelverluste. Denn ihr Zweck war vom Menschen verzehrt zu werden. 
 
 

\paragraph{Nachernteverluste}
Die Nachernteverluste betragen ca. 1,593 Mio. t, das Vermeidungspotential ist gering. %TODO (WWF das große wegschmeißen, s.10)

Nachernteverluste werden bei pflanzlichen und tierischen Nahrungsmitteln unterschieden. Bei pflanzlichen Lebensmittel können Verluste durch den Transport und die Lagerung entstehen.  Tierischen Produkte verzeichnen einen Verlust wenn das Tiere während des Transportes zum Schlachthof stirbt. Eiern und Milch zählen ebenfalls zu den Nachernteverluste wenn sie während des Transportes zum Abnehmer verderben gemeint. %TODO (Gustavsson et al., 2011).
 
Im internationalen Vergleich sind die Nachernteverluste innerhalb Deutschland relativ gering. Zurückzuführen ist das auf die hohen technologischen Standards bei Lagerung und Transport sowie infrastrukturelle Voraussetzungen. %TODO (vgl. iSuN, 2013).

\subsubsection{Lebensmittelverarbeitung}
Die Verluste in der Lebensmittelverarbeitung belaufen sich auf ca. 2,61 t, ein Zehntel davon wäre vermeidbar. %TODO (WWF das große wegschmeißen, s.10)

Tierische und pflanzliche Lebensmittel, die während der industriellen
oder häuslichen Weiterverarbeitung aussortiert werden, zählen zu Verlusten in der Lebensmittelverarbeitung.%TODO $(Quelle: https://www.wwf.de/fileadmin/fm-wwf/Publikationen-PDF/WWF_Studie_Das_grosse_Wegschmeissen.pdf, S. 31). $  

Die Ursachen für Verluste in diesem Segment sind vielfältig: 
\begin{itemize}
  \item Qualitätssicherung 33 Prozent: Der Verlust der Ware an dieser Stelle entsteht durch intern festgelegte Qualitätskriterien. Ein Beispiel dafür wären beschädigte Verpackung, da diese anfällig für Veränderungen durch Fremdkörper sind oder es zur mikrobiologischer Belastung kommen kann. Darüber hinaus existiert durch die VO (EG) Nr. 852/2004 die Verpflichtung Proben und Rückstellmuster im Zuge der Qualitätssicherung anzulegen. Mit diesen sichern sich die Hersteller ab, dass ihre Ware in einem unbedenklichen Zustand ist. Die Proben werden anschließend entsorgt. %TODO (Kranert,et al., 2012, S. 24, 207), (Escaler & Teng, 2011), (Göbel, et al., 2012, S. 35-38).
  \item Technische Störungen 29 Prozent: Technischen Störungen oder auch das Nutzen von veralteter Technik führen zu Ausfällen in der Produktion oder Fehlern im Herstellungsprozess. Dadurch entstehen Fehlchargen die sich nicht für den Verkauf eignen %TODO (Göbel, et al., 2012, S. 40), (Kranert, et al., 2012, S. 207 ff.). 
  Auch falsch ettikierte, beschädigte oder fehlerhafte Ware werden entsorgt, da durch eine neue Etikettierung oder VerpAckung zu hohe Kosten entstehen. %TODO (Kranert, et al., 2012, S. 207 ff.)
  \item Beschädigung und Verderb 18 Prozent: bei Transport, Verpackung, Lagerung, etc. %TODO (Kranert, et al., 2012, S. 207 ff.)  
  \item Überproduktion 17 Prozent: Durch Fehlkalkulation oder Planungsfehler entstehen Überschüsse %TODO(Kranert, et al., 2012, S. 207). 
 Diese Lebensmittel, werden trotz einwandfreien Zustandes entsorgt. Da die Lagerkapazitäten beschränkt sind und aus Produzentensicht die Lagerkosten für die Überproduzierte Ware zu hoch sind. %TODO (Göbel, et al., 2012, S. 32) 
 Retourware kann ebenfalls für Überbestände sorgen. Gründe dafür können z. B. Stornierung aufgrund von mangelndem Interesse sein oder wenn die vertraglich festgelegte Haltbarkeit nicht eingehalten wurde. %TODO(Göbel, et al., 2012, S. 38).  
  \item Sonstiges 3 Prozent. \end{itemize}

%TODO (Kranert et al 2012). 
 
Trotz der vielfältigen Ursachen sind die vermeidbaren Lebensmittelverluste in diesem Segment relativ gering. Resultierend aus den  verlustgeminderten und kosteneffizienten technologischen Standards die innerhalb von entwickelten Industriestaaten wie Deutschland gelten. %TODO (wwf das große wegschmeißen,. S.32) 


\subsubsection{Groß- und Einzelhandel}
Im Groß- und Einzelhandel betragen die Lebensmittelverluste 2,575 Mio. t, Vermeidungspotenzial bis zu 90 Prozent. %TODO (WWF das große wegschmeißen, s.10)

Abfälle im Handel entstehen unter anderem durch die Bestrebung der Supermärkte und Discounter, dem Kunden jederzeit ein breites Sortiment anbieten zu können. Die Erwartungshaltung der Kunden an Frische und Optik spielen dabei ebenfalls eine Rolle. Aber auch das wenig vorhersagbare Einkaufsverhalten des Kunden, trägt seinen Beitrag zu Verlusten im Handel bei. %TODO (Quelle: Fakt ist Lebensmittelverschwendung S.7). 
Dennoch gibt es vielfältige andere Gründe, die in diesem Kapitel aufgeführt werden.


\paragraph{Lieferkette und Distribution}
Lebensmittelabfälle entlang der Lieferkette, haben mehrere Ursachen: fehlerhafte Abstimmungen zwischen Händler und Lieferanten, Unterbrechung der Kühlkette, mangelhafte Sicherung der Paletten und Beschädigung der Transport- und/oder Produktverpackung. Des Weiteren können unkalkulierbare Ereignisse z. B. Stromausfälle, Unfälle während der Lieferung und/oder Lagerung der Ware eintreten die zur Beschädigung der Ware führt. %TODO(Göbel, et al., 2012, S. 41), (Kranert, et al., 2012, S. 2)
	 Besonders anfällig sind Fleisch und Wurstwaren, Fisch und Meeresfrüchten oder Milchprodukte, da sie sehr temperaturempfindlich sind. %TODO (Göbel, et al., 2012, S. 41), (Kranert, et al., 2012, S. 2)
	 \paragraph{Warenpräsentation}
Wie bereits aufgeführt ist Verfügbarkeit einer breiten Produktvielfalt zu jeder Tages- und Jahreszeit das Bestreben des Handels, aber auch die Erwartungshaltung des Konsumenten.%TODO (Kranert, et al., 2012, S. 208). 
Dies führt dazu das bis Ladenschluss Regale und Frischetheken (z.B. Backwaren) immer wieder aufgefüllt werden. (Göbel, et al., 2012, S. 39). Um dies zu gewährleisten, bezieht der Einzelhandel mehr Ware als gekauft wird. %TODO (SRU, 2012, S. 185)

\paragraph{Planung und Organisation}
Trotz EDV gestützter Logistikabläufe ist es für den Handel schwierig, die Nachfrage für Produkte zu kalkulieren. (Kranert, et al., 2012, S. 222). Die Ursachen dafür können beispielsweise bevorstehende Feiertage oder Wettereinflüsse sein und erstrecke sich auf alle Lebensmittelkategorien. Kritisch sei dabei Saisonware, die für einen bestimmten Anlass produziert wurde z.B Weihnachten, Ostern oder Fußballmeisterschaften. Auch ein Redesign von Verpackungen kann dazu führen das Restbestände entsorgt werden. (Priefer & Jörissen, 2012, S. 35-37) %TODO quelle und nochmal lesen


\paragraph{Convenience und Fresh-Cut}
Innerhalb der letzten Jahre hat sich eine wachsende Nachfrage an schnell verderblichen Lebensmitteln und verzehrfertigen Produkten(Convenience und Fresh-Cut) entwickelt. Dadurch das diese Produkte bereits verzehrfertig sind(geschält, gemischt oder klein geschnitten), werden sie anfälliger für Verderb.%TODO (HLPE, 2014, S. 46-47), (Göbel, et al., 2012, S. 29 u. 39)
Das MDH ist auf Stunden begrenzt und daher wird diese Ware noch häufiger entsorgt. %TODO (Teitscheid, 2014)

\paragraph{Mindesthaltbarkeitsdatum}
Konsumenten greifen bei der Auswahl im Handel nach frischesten Ware. Daraus resultiert das die verkaufsfähigkeit von älterer Ware gering ist. (Kranert, et al., 2012, S. 208) 
Selten gibt es finanzielle Anreize aber auch die nötigkeit, Ware die nah am MHD bzw. VBD ist zu kaufen.   (Göbel, et al., 2012, S. 38), (Priefer & Jörissen, 2012, S. 38). Ware mit abgelaufenen MHD muss von dem Händler nicht aus dem Verkauf genommen werden, wenn wer dies aber nicht tue geht die Verantwortung für einen unbenklichen Verzehr auf ihr über.(Waskow, 2013, S. 274) Da diese Verantwortung vom Handel nicht getragen werden möchte wird die Ware aus dem Verkauf genommen.(Göbel, et al., 2012, S. 40).


\subsubsection{Großverbraucher}
Die Lebensmittelverluste bei Großverbraucher betragen 3,4 Mio. t vermeidbar wäre davon 70 %. %TODO (WWF das große wegschmeißen, s.11)

Unter dem Begriff Großverbraucher werden verschiedenste Außer-Haus-Bewirtungsstätte bezeichnet. Darunter fallen das Gaststätten- und Behebungsgewerbe, Betriebsverpflegung, Alten- und Pflegeheime, Schule und anderen Bildungseinrichtungen, Krankenhäuser, Kinderbetreuungseinrichtungen sowie die Bundeswehr. Nach Kranert et al. (2012) fallen bei den Großverbrauchern in der Gastronomie mit etwa 1,0 Mio. t jährlich die größte Menge Abfall an, danach folgt die Betriebsverpflegung mit 275.000. %TODO seite nennen

\paragraph{Hygiene und Sicherheitsvorschriften}
Eine Ursache für Lebensmittelabfälle bei Großverbrauchern sind verschiedene Hygiene und Sicherheitsvorschriften die eingehalten werden müssen.

\begin{itemize}
  \item Verfütterungsverbot: Die Verordnung (EG) Nr. 999/2001 Artikel 7 Absatz 1 und 2 besagt, dass Lebensmittelabfälle tierischen Ursprungs nicht an Wiederkäuer und weitere Nutztiere verfüttert werden dürfen.
  Dies geschieht aus Seuchenhygienischen Gründen (BSE), dennoch können so Lebensmittelabfälle nicht weiterverwendet werden. %TODO (quelle: https://www.lanuv.nrw.de/verbraucherschutz/lebensmittelsicherheit/futtermittel/betriebslisten/verfuetterungsverbot). 
  
  \item Die Weitergabe von Resten der zubereiteten Mahlzeiten ist aus hygienischen Gründen nur zulässig, wenn die Speisen die Küche noch nicht verlassen haben. %TODO(Kranert, et al., 2012, S. 218)
  \item Original verpackte Lebensmittel die bereits an Kunden ausgegeben wurden, müssen entsorgt werden. Das liegt zu einem an hygienischen Bedenken und zum anderen daran das zum einen die Kühlkette unterbrochen wurden (z. B. Butter in Portionsverpackungen) %TODO (Kranert, et al., 2012, S. 218)
  \item Verantwortlichkeit der Gastronomen: Eine Lebensmittelvergiftung hätte nicht nur rechtliche Konsequenzen, sondern würde auch dem Ruf schaden, daher sind sie sehr vorsichtig und gehen kein Risiko ein.  %TODO (Kranert, et al., 2012, S. 218)
  \item Küchenabfälle aus international eingesetzten Verkehrsmittel: Diese dürfen nach der Verordnung (EG) Nr. 1069/2009 nicht weiter verwendet werden und werden verbrannt. Auch, wenn es sich original verpackte Lebensmittel handle.
\end{itemize}

\paragraph{Verluste im Bereich der Küche}
Innerhalb einer Großküche kommt es durch verschiedene Ursachen zur Lebensmittelverschwendung: 
\begin{itemize}
  \item Fehler in der Küche d.H. angebrannt, versalzen.  %TODO Quelle
  \item Falsche Lagerung: In Großküchen kann aufgrund der Menge der auf Vorrat gehalten Lebensmittel schnell die Übersicht verloren gehen. Gerade, wenn frische Lebensmittel, vor den bereits vorhanden eingeräumt werden und die älteren Lebensmittel dadurch übersehen werden und dadurch verderben.  %TODO(kranert)
  \item Nichtverwendung von übrig gebliebenen Speisen: Außerhalb der Hygienevorschriften ist es häufig dem Küchenpersonal nicht möglich übrig geblieben Speisen erneut einzusetzen. Eine Untersuchung von %TODO
  Müller ergab, dass rund die Hälfte der bereits zubereiteten Speisen, die nicht in die Ausgabe gelangt sind, erneut verwendet werden könnten. Gründe dafür sind laut dem Küchenpersonal Platz- und Zeitmangel, übrig geblieben Speisen können nicht rechtzeitig fachgerecht abgekühlt und verpackt werden. %TODO(Engström) 
  Aber auch die Einhaltung Fixen Menüpläne lasse keine zusätzlichen Speisen bzw. Mahlzeiten vom Vortag zu.
   \item Mangelndes Bewusstsein zum Aufkommen der Speiseabfälle: Durch die Arbeitsteilung in professionell geführten Küchen in Restaurants fehlt den Mitarbeitern die Einblicke auf die Menge der Speisereste. Beispielsweise kontrollieren Mitarbeiter die für die Planung der Anzahl der Speisen, in der Zubereitung oder Ausgabe tätig sind nicht die Menge der Essensreste und können daraus keine Rückschlüsse ziehen. Bei einer Befragung in Verpflegungseinrichtungen von Frübis u. Class gaben 3/4 der Einrichtungen an, die ausgegeben Teller zu kontrollieren und Portionsgrößen dynamisch anzupassen bei zu hohen Rückläufen. [45] %TODO(Kranert)

  \item Schwierige Nachfrage Kalkulation: Da es häufig keine Daten zur Anzahl der täglichen nachgefragten Portionen gibt, neigt das Personal dazu großzügig zu kalkulieren um notfalls mehr Kunden versorgen zu können. Zusätzlich ergab eine Befragung von 353 Einrichtung von Kürbis u. Class das die Küchenleitung sich bei der Kalkulation größtenteils auf ihrer Erfahrung (55 Prozent) stützen und 45 Prozent auf die Zahl der tatsächlich Anwesenden zurückgreifen können. %TODO (kranert)

\end{itemize}

\paragraph{Verluste beim Kunden}
Neben den bereits aufgeführten Ursachen,kommt es auch beim Kunden selbst zu Nahrungsmittelverlusten:


\begin{itemize}
  \item Menüauswahl: Wenn in einem Lokal wenig verschiedene Menüs bzw. Speisen angeboten werden, ist eine Annäherung an die individuellen Bedürfnisse des Kunden schwierig. So kann es immer wieder passieren, dass die Tellerreste durch ungewollte Zusätze im Menü entstehen. Auch die fehlende Möglichkeit sich Speisen selbst zusammenzustellen, wirkt sich auf die Tellerreste aus. %TODO (vgl. [45], [84])

  \item Portionsgrößen: Eine der meist angegeben Ursachen für Lebensmittelabfälle innerhalb deutscher Gemeinschaftsverpflegung ist die Größe der Portion. Oft werden die Mahlzeiten händisch portioniert. Häufig führt dies zu größeren Portionen, als wenn genormte Schöpfkellen verwendet werden. In diesem Zusammenhang fanden Frübis u. Class heraus das mit zunehmender Portionsgröße auch die Tellerreste bzw. Speiseabfälle zunahmen. [45] %TODO
  \item Art der angebotenen Portionierung: Das Vorportionierung der Speisen ist eine weitere Ursache für das Aufkommen für Tellerreste. [45], [40] Denn, wenn Kundenwünschen bezüglich Austausch oder Ausschluss von Teilkomponenten eines Gerichtes sowie die Größe der Portion nicht berücksichtigt werden können. %TODO (kranert)
  \item Buffets: Auch wenn im vorangegangen Abschnitt das selber kalkulieren und zusammenstellen von Speisen durchaus Vorteile aufweist, ist zu bedenken das Speisen in Buffetform auch zur Lebensmittelverschwendung beitragen. Dies liegt daran, dass die Erwartung der Konsumenten ist das bei einem Buffet alle Speisen in großer Menge vorrätig sind und die Auslage nicht abgesessen aussieht. Des weiten führen die einheitlichen Büffetpreise dazu, das Gäste dazu neigen mehr Nahrung nehmen, also ihnen möglich ist zu verzehren.  %TODO (Monier, et al., 2010, S. 39)
  \item  Informationsdefizite: Thorwarth u Geißler konnten im Rahmen ihrer Untersuchung diese Ursache bei verschiedenen Fleischabfällen in einer Justizvollzugsanstalt feststellen. Konnte ein Häftling nicht erkennen, ob er eine bestimmte Speise in Hinsicht seiner religiösen Gesinnung konsumieren könnte oder nicht, wurde diese unter Umständen nicht gegessen. Dies gilt ebenfalls für Allergiker, wenn bei Speisen oder -komponenten deren Unbedenklichkeit nicht erkennbar war. %TODO(kranert)  \item 
  
  \paragraph{Problematik Messbarkeit}
Die Problematik bei der Abfallmenge im Bewirtungssektor ist das keine gesicherten Angaben getätigt werden können. Da die Betriebe in Eigenverantwortung die gewerblichen Abfälle durch ein Privatunternehmen entsorgen lassen. Des Weiteren ist es möglich, dass gastronomische Kleinbetriebe die ö.r. Biotonne für ihre Speiseabfälle verwenden, auch wenn dieses nicht gestattet ist. %TODO $(quelle : https://www.umweltbundesamt.de/sites/default/files/medien/461/publikationen/4010_0.pdf)$

\end{itemize}


\subsubsection{Privathaushalte}
In Privathaushalten werden jährlich 6,7 Mio. t Lebensmittel weggeschmissen, dies entspricht 70 Prozent des gesamten Lebensmittelverlust. %TODO (WWF das große wegschmeißen, s.11)



Lebensmittelverluste in Privathaushalten beziehen sich auf bereits gekaufte und verzehrfertige Nahrungsmittel die entsorgt werden. %TODO (WWF das große wegschmeißen, s.37)
Die Ursachen für Lebensmittelverschwendung in privaten Haushalten sind nicht nur individuell, sondern auch ein gesellschaftliches Problem. Die Einwohner in Deutschland leben in einer Konsum-, Überfluss- und Wegwerfgesellschaft. Neben Lebensmitteln betrifft das auch weitere Konsumgüter, die bereits vor der Ende ihrer Lebensdauer entsorgt werden. %TODO(quelle Krantert S 217 [184].

\paragraph{Mangelnde Einkaufsplanung}
Ohne Vorbereitung Einkäufe zu tätigen, führt zu Spontankäufen getrieben von Hunger und Optik. Dabei werden Lebensmittel, die sich bereits im häuslichen Vorrat befinden vergessen und doppelt gekauft. %TODO (quelle Fakt ist S.8)

\paragraph{Lagerung}
Den Konsumenten sind oft die unterschiedlichen Ansprüche der Lebensmittel nicht bewusst. Gerade bei Obst und Gemüse wird unterschiedlich gelagert z. B. Hell, dunkel, kühl oder Zimmertemperatur. Ein weiterer Fehler bei der Lagerung ist, neu eingekaufte Lebensmittel vorne vor die älteren Lebensmittel einzuräumen. Die alten Lebensmittel werden vergessen, verderben und werden entsorgt. %TODO (quelle Webseite: zu gut für die Tonne)

\paragraph{Zubereitung}
Während der Zubereitung von Speisen sind es verfahren wie Schälen oder Putzen die zur Lebensmittelverschwendung führt. Aber auch persönliche Vorlieben sind eine Ursache z. B. Brot ohne Rinde, obwohl diese essbar ist. Ein weiterer Faktor sind mangelnde Kenntnisse über die Weiterverwendung von Essensreste, die obwohl noch genießbar, oft weggeworfen werden. %TODO (quelle Webseite: zu gut für die Tonne) 

\paragraph{Portionen/Mengen}
Die im Einzelhandel angebotenen große Verpackungen können meist von älteren Personen (60+) sowie Ein Personen Haushalten nicht vor verfall konsumiert werden. (Studie GFK) Gefördert wird diese Menge Problematik durch die Tatsache, dass „günstige“ Sonderangebote nur in größeren Packungseinheiten oder Multi-Packs angeboten werden. 

In Personen mit mehr als drei Personen steigt hingegen die Menge an zu viel Gekochtem Essen bzw. Tellerresten. Erklärungen hierfür stellen sicherlich die verschiedenen Geschmäcker in einer größeren Familie, oder auch der bekannte Futterneid bei Kindern (den Teller zu vollladen) dar. 


\paragraph{Mindesthaltbarkeitsdatum}
Das Mindesthaltbarkeitsdatum wird oft falsch vom Konsumenten interpretiert, dadurch werden Lebensmittel die noch verzehrbar sind zu früh entsorgt. Hinzukommend sind Haltbarkeitsdaten oft nicht eindeutig oder inkonsequent bzw. irreführend angegeben. %TODO $(quelle: https://www.wwf.de/fileadmin/user_upload/studie_tonnen_fuer_die_tonne.pdf s.23) $

\paragraph{Mangelnde Wertschätzung}
Im Jahre 1950 betrugen die monatlich Ausgaben für Nahrungs- und Genussmittel rund 50 Prozent des Haushaltsbudgets. [Kranert 186] 60 Jahre später werden in Deutschland nur noch 9,5 Prozent des monatlichen Einkommens für Nahrungsmittel und (alkoholfreie) Getränke ausgeben. [22] Daraus lässt sich schlussfolgern das Lebensmittel deutlich günstiger geworden sind. Das heutige Überangebot an Nahrungsmittel steht darüber hinaus im Zusammenhang mit der immer weiteren Ausbreitung von Adipositas und anderen Essstörungen. %TODO [188].


Die geringe Wertschätzung für Lebensmittel hängt auch mit der zunehmenden Entfremdung gegenüber Herstellung und Verarbeitung der Lebensmittel. Der Konsument kennt häufig nicht den Ursprung oder die Herkunft der Lebensmittel. Auch emotionale Bildung zu Lebensmitteln ist verlorenen gegangen, durch das nicht weiterführen verschiedenen Familientraditionen und das Fehlen von gemeinsamen Mahlzeiten durch Entsynchronisierung der Tagesabläufe der Haushaltsmitglieder. %TODO Krantert


Hinzukommend trägt der starke Preiswettbewerb im Lebensmittel Handel sein rest bei. Vorallem verursacht durch die starke Position der Discounter. Durch die geringen preise, erhalten Lebensmittel ein wertmäßig geringe Bedeutung für den konsumenten. Hinzukommend zu dem Preisbewusstsein, steigen steigt die Anforderungen an an Optik und Geschmack. %TODO (studie verringerung von lebensmittelabfällen )


\subsection{Einfluss der Politik}
Nachdem die Ursachen von Lebensmittelverschwendung entlang der Wertschöpfungskette dargelegt wurden, wird ein Blick auf den Einfluss der Politik geworfen. Denn das Bundeskabinett hat sich als Ziel gesetzt bis 2030 die Zahl der Lebensmittelabfälle zu halbieren. %TODO $(https://www.bundesregierung.de/breg-de/aktuelles/lebensmittelabfaelle-halbieren-1581854) $

Dieses Ziel soll mit unterschiedlichen Projekte und Initiativen erreicht werden: 
\begin{itemize}
  \item Zu gut für die Tonne: Das Bundesministerium für Ernährung und Landwirtschaft hat im März 2012 den ersten Grundstein mit der Initiative “Zu gut für die Tonne” gelegt %TODO (Quelle BMEL). 
  Diese informiert über Lebensmittelwertschätzung, die Ursachen der Verschwendung und wie diese reduziert werden kann. Auf dem Webauftritt sind darüber hinaus Rezepte für Speisereste und weitere hilfreiche Tipps zu finden. %TODO $(https://www.bundesregierung.de/breg-de/aktuelles/lebensmittelabfaelle-halbieren-1581854)$

  \item “Zu gut für die Tonne” - Bundespreis: Seit 2015 zeichnet das BMEL Projekte und ihre Initiatorinnen und Initiatoren mit dem Zu gut für Tonne” – Bundespreis aus um ihr Engagement gegen Lebensmittelverschwendung zu würdigen und weiter zu unterstützen. %TODO $(https://www.bundesregierung.de/breg-de/aktuelles/lebensmittelabfaelle-halbieren-1581854)$

  \item www.lebensmittelwertschaetzen.de: Diese Webpräsenz wird von der Bundesregierung und den Ländern dazu genutzt, um Initiativen gegen Verwendung zu veröffentlichen. Bereits 100 Projekte finden sich auf der Plattform wie z. B. die "BrotRetter". %TODO$(https://www.bundesregierung.de/breg-de/aktuelles/lebensmittelabfaelle-halbieren-1581854) $
   Eine Initiative bei der das Brot vom Vortag zu einem günstigeren Preis in den eigenen Geschäften angeboten wird. Alternativ wird das Brot da Tafeln oder landwirtschaftliche Betriebe für Tierfutter weitergegeben. %TODO $(https://www.jb.de/das-unternehmen-junge/nachhaltigkeit/brotretter/)$
   

  \item Förderung von technischen Innovation: Die Bundesregierung fördert beispielsweise Apps wie Beste Reste oder Too Good to go im Kampf gegen die Lebensmittelverschwendung. .%TODO $(https://www.bundesregierung.de/breg-de/themen/digitalisierung/apps-lebensmittelverschwendung-1653238)$

  \item Lebensmittelabfälle in Deutschland – Baseline 2015: Die ́Nationale Strategie zur Reduzierung der Lebensmittelverschwendung ́ vom Bundeskabinett, im Februar 2019 verabschiedet, sieht vor das in interministeriellen Arbeitsgruppen eine bundeseinheitliche Bilanz der Lebensmittelabfälle in Deutschland erarbeitet. Als Unterstützung sollte das Thünen-Institut in Kooperation mit der Universität Stuttgart einen Vorschlag für die Baseline 2015 als Disskusionsgrundlage vorbereiten. Dabei werden die Lebensmittelabfälle der gesamten Wertschöpfungskette berechnet um eine Grundlage für die kontinuierliche Berichterstattung in den Jahren 2020 bis 2030 zu besitzen. %TODO (quelle https://www.thuenen.de/media/publikationen/thuenen-report/Thuenen_Report_71.pdf , s. XI)
)\end{itemize}


\subsection{Umwelteffekte Lebensmittelverlusten und
-verschwendung}
\subsubsection{Flächenfußabdruck}
Durch die Ernährungsweise der Menschen werden nicht nur natürliche Ressourcen in Anspruch genommen, sondern auch Fläche.

Die Fläche Deutschland erstreckt sich über ein Territorium von ca. 35,7 Mio. ha (oder 357.000 km2) (Destatis, 2014b). Landwirtschaftliche Nutzfläche davon sind 44 Prozent bzw. 16,8 Mio ha. Als Ackerfläche werden davon knapp 11,9 Mio. ha (Destatis, 2014d) genutzt. %TODO Quelle

Da der Bedarf an Agrarprodukten mit der Fläche, der Deutschland zur Verfügung steht nicht gedeckt werden kann, wird zusätzlich Fläche im Ausland genutzt. Nach Berechnung der WWF Studie „“ beläuft sich diese Fläche auf 5,5 Mio. ha, der sich aber als Bilanzwert versteht. Der Großteil der im Ausland genutzt Fläche liegen in Südamerika mit fast 2,8 Mio. ha und Brasilien mit 1,5 Mio. ha. Die restliche Fläche wird bei deren Mitgliedsstaaten der EU 1,9 Mi. ha), Nordamerika (ca. 0,7 Mio. ha), Asien (mehr als 0,5 Mio. ha), Subsahara-Afrika (knapp 0,4 Mio. ha) und Ozeanien (fast 0,7 Mio. ha) abgezogen. %TODO Quelle


Damit steht Deutschland eine Fläche von rund 21,659 Mio. ha für landwirtschaftliche Nutzung zur Verfügung. Abzuziehen von der Fläche sind Agrarprodukte die nicht als Nahrungsmittel dienen, dann verbleibt eine Fläche von 19,369 Mio. ha die für die Ernährung der Einwohner Deutschland benötigt wird. Pro Einwohner entspricht das ein Flächenfußabdruck von 2.397 m2. Davon lassen sich 70 Prozent der Fläche auf Nahrungsmittel mit tierischem Ursprung zurückzuführen. Pro Kopf auf alle Einwohner bezogene fallen auf unseren Fleischkonsum 1.019 m2, Milchprodukte 602 m2. Darauf folgen Getreideerzeugnisse in Hohe von 231 m2. %TODO Quelle

%TODO Seite $(quelle: https://www.wwf.de/fileadmin/fm-wwf/Publikationen-PDF/WWF_Studie_Nahrungsmittelverbrauch_und_Fussabduecke_des_Konsums_in_Deutschland.pdf) $
Um den Flächen Fußabdruck zu verringern ist neben einem geringeren Fleischverzehr und der Bevorzugung von regionalen und saisonalen Produkten auch der Umgang mit den Nahrungsmitteln wichtig. Denn durch die Vermeidung von Nahrungsmittelverluste, kann bis zu 2,4 Mio. ha Ackerfläche eingesparte und anderweitig genutzt werden. Umgerechnet entspricht dies in etwas ein Fläche Mecklenburg-Vorpommerns. %TODO Quelle

\subsubsection{Klimafußabdruck}
Lebensmittelverschwendung ist auch Belastung für das Klima, denn rund 40 Prozent der jährlichen Pro-Kopf-Emission in Deutschland werden durch Ernährung und Konsum verursacht. %TODO  $ (https://www.umwelt.nrw.de/fileadmin/redaktion/Broschueren/lebensmittelverschwendung_bf.pdf)$ 

THG-Emissionen im Ernährungskontext

Um THG-Emission berechnen zu können, bedarf es einen Standard. Meist wird dafür der Product Carbon Footprints %TODO Quelle(vgl. u. a. Plassman et al., 2010)
verwendet. Alle Emissionen, die bei einem Produkt entlang der Wertschöpfungskette entstehen werden addiert. Bezogen auf die landwirtschaftliche Wertschöpfungskette, sind das alle THG-Emissionen die z. B. durch die Produktion, Nutzung von Böden oder Veredlung und Entsorgung von Lebensmitteln. %TODO Quelle

Aktuell beträgt die THG-Emission unserer Ernährung 1.991 kg pro Kopf bzw. 161. Mio. für ganz Deutschland. t CO2-Äquivalenten. Das entspricht mehr als das Doppelte an THG-Emissionen aus Industrieprozessen oder der Landwirtschaft in Deutschland. %TODO Quelle  

Tierische Erzeugnisse (inklusive Fisch) sind dabei für 63 Prozent der Emission verantwortlich. Besonders hervorzuheben ist dabei Fleisch, dies macht mit 723 kg CO2-Äquivalente rund 36 Prozent aller THG-Emissionen unserer Ernährung aus.  %TODO Quelle

Unsere aktuellen direkten und indirekten THG-Emissionen, die durch den Nahrungsmittelverbrauch entstehen, können durch signifikanten Änderungen unsere Ernährung und der Vermeidung von Lebensmittelabfällen bis 2050 um 23 Prozent verringert werden. %TODO Quelle


\section{Lösungsansätze zur Verhinderung von Lebensmittelverschwendung in Privat-Haushalten}
\subsection{Verderb erkennen}
Der erste Schritt um Lebensmittelverluste zu verringern ist, den Verderb dieser richtig erkennen zu können. Denn meist verlässt sich der Konsument ausschließlich auf das Mindesthaltbarkeitsdatum.

 Doch wie verderben Lebensmittel? Die Ursachen können biochemisch, chemisch, physikalisch oder mikrobiologisch Veränderungen sowie der Befall von Schädlingen sein. %TODO Quelle
\subsubsection{Verderb durch Mikroorganismen}
Die Hauptursache für Lebensmittelverderb sind mikrobiologisch Veränderungen. Damit sind Schimmelpilze, Bakterien oder Hefen gemeint. Diese gelangen durch Tiere, Menschen oder verunreinigte Gegenstände(z.B. unsauberes Messer) über die Luft auf die Lebensmittel. Wenn günstige Bedingungen(Temperatur, Wasseraktivität) gegeben sind, vermehren sich die Mikroorganismen auf den Lebensmitteln explosionsartig. Bei der Ausbreitung bauen die Mikroorganismen Inhaltsstoffe des Lebensmittel ab und scheiden Stoffwechselprodukte aus. Dies hat zu Folge das die befallenen Lebensmittel sauer werden, faulen, gären oder schimmeln.  %TODO Quelle

\subsubsection{Physikalische Ursachen}
Physikalische Faktoren wie Zeit, Temperatur, Lichtstärke, Wassergehalt, Luftfeuchtigkeit oder Druck beeinflussen den Verderb eines Lebensmittels. Beispielsweise schimmelt Brot, wenn es zu Warm oder feucht gelagert wird, wohingegen bei zu kühler Lagerung es altbacken wird. %TODO Quelle

\subsubsection{Biochemische und chemische Veränderungen}
Bei Lebensmittelverderb durch Biochemischen bzw. Chemischen Reaktionen, spielen die Enzyme, die in Lebensmittel enthalten sind eine Rolle. Mit Enzymen sind Proteine gemeint, die innerhalb der Zellen von Pflanzen und Tieren Stoffe biochemisch ab- und umbauen. Während des Verberbes werden durch die Enzyme Farb- und Aromastoffe oder Vitamine abgebaut. Bei längerem Kontakt mit Sauerstoff können beispielsweise Fette und Öle ranzig werden. %TODO Quelle
\subsubsection{Befall mit tierischen Schädlingen}
Mäuse, Motte oder Käfer tragen ebenfalls zum Verderb von Lebensmitteln bei. Bei Kontakt mit diesen übertragen die Schädlinge Schmutz, Mikroorganismen und Pilzsporen. %TODO Quelle

\subsubsection{Nutzen der Sinnesorgane}
Der Verderb einige Lebensmittel lässt sich einfach mit den Sinnesorganen herausfinden: Bei Milch und Milchprodukte gibt bereit eine Veränderung der Verpackung (Joghurtbecher dehnt sich aus beim Verderb), der Geruch oder Geschmack Auskunft über die Frische des Produktes.
Vorsicht ist geboten bei gekühlter Fertigware z. B. Würstchen, Frikadellen oder Teigware. Meist können die krankmachenden Keime mit dem bloßen Auge nicht erkannt werden.  %TODO Quelle

Ein erster Schritt Lebensmittel Verderb zu Hause zu verhindert die optimale Lagerung die im nächsten Kapitel deutlich erklärt wird. %TODO Quelle


\subsection{Lagerung von Lebensmitteln}
\subsubsection{Lagerungsorte}
\paragraph{Kühlschrank}
Zu unterscheiden sind Kühlschränke mit Verdampfer, bei dem es innerhalb des Kühlschrankes zu Temperaturunterschieden, durch die eine natürliche Luftzirkulation, kommt. Hingegen bei einem Kühlschrank mit Umluftkühlung sorgt ein Ventilator dafür das die kühle Luft gleichmäßig verteilt wird. Dies führt zu geringen Temperaturunterschieden.
Das bedeutet während bei der Umluftkühlung die Lebensmittel beliebig eingeräumt werden können, muss bei einem Kühlschrank mit Verdampfer die Temperaturunterschiede bedacht werden.
Die kälteste Stelle, die untere Ablagefläche beispielsweise ist Ideal für leicht Verderbliches z. B. Fleisch. Darüber können Milch und Milchprodukte, Käse sowie der Inhalt geöffneter Konserven aufbewahrt werden. Die Türfächer haben die höchste Temperatur, daher sollten dort nur Lebensmittel gelagert werden, die leicht kühl bedürftig sind z. B. Getränke, Butter, Eier, Marmeladen, Dressings, Soßen und Tuben. Das Obst- und Gemüsefach ist ebenfalls etwas wärmer als die durchschnittliche Innentemperatur. Alle Kühlschrank tauglichen pflanzlichen Lebensmittel werden hier gelagert. %TODO Quelle
(https://www.bzfe.de/inhalt/lebensmittel-richtig-lagern-645.html)

Neben der richtigen Lagerung ist auch die regelmäßige Reinigung des Kühlschranks wichtig. Ebenfalls sollten faulen Lebensmittel direkt entsorgt werden, damit sich die Bakterien nicht weiter verbreiten können. Angebrochenen Lebensmittel im Kühlschrank, deren Verpackung sich nicht mehr schließen lässt, sollte zudem schnellstmöglich in ein Aufbewahrungsgefäß gefüllt werden um Verderb und Verunreinigung des Kühlschranks zu vermeiden. %TODO Quelle (quelle: https://www.zugutfuerdietonne.de/service/presse/pressemitteilungen/frische-im-kuehlschrank-alles-eine-frage-der-hygiene/) 
\paragraph{Speisekammer}
Die Speise bzw. Vorratskammer, dient dazu Lebensmittel Trocken und Dunkel bei durchschnittlich 15 bis 20 °C zu lagern. Zu beachten dabei ist ebenfalls dem First in First out regel beim Einräumen der Ware, abgebrochen Packungen in Glas, Metall oder Kunststoff Behälter zu füllen die sich luftdicht verschließen lassen. Am besten lassen sich in der Speisekammer Mehl, Salz, Zucker, Vollkonserven, Trockenprodukte wie Reis, Nudeln oder Cerealien lagern. %TODO Quelle

\paragraph{Gefrierschrank}
Der Gefrierschrank wird in der Regel verwendet, um verschiedene Tiefkühlwaren aufzubewahren. Aber auch selbst gekochte Gerichte, sowie frisches Gemüse oder Obst können im eigenen Haushalt eingefroren werden.
Der Prozess des Einfrierens wird im Kapitel “Verlängerung der Haltbarkeit” näherer erläutert. In diesem Abschnitt geht es nur im Lagerungsmöglichkeit des Gefrierschrankes bzw. -truhe oder -fach. %TODO Quelle


\subsubsection{Lagerung nach Lebensmittelkategorie}
\paragraph{Brot und Backwaren}
Um Haltbarkeit von Brot zu verlängern, muss dieses vom Austrocknen geschützt werden. Am besten eignet sich, statt der Kunststofftüte, in der es meistens verpackt ist, ein Brotkasten oder Tontopf mit Deckel. Aber auch eine zu feuchte Umgebung mit wenig Luftzirkulation, kann sich negativ auf die Haltbarkeit auswirken. Besonders Schnittbrot ist anfällig dafür Wasser anzusammeln, was Angriffsfläche für Schimmel bietet. Um das Brot länger frisch zu halten, sollte es am besten im Stück gekauft werden und in einem regelmäßig gereinigten Brotkasten aufbewahrt werden. Alternativ lassen sich Brot und Brötchen einfrieren um zu einem späteren Zeitpunkt aufgebacken zu werden. %TODO Quelle

 \paragraph{Milchprodukte}
Der Großteil der Milchprodukte werden im Kühlschrank aufbewahrt. Am besten eignet sich der mittlere Bereich des Kühlschrankes zur Aufbewahrung dieser. Bei der Lagerung von Käse ist zu beachten, das jeder Käse separat verpackt werden sollte, um Schimmel zu vermeiden. Um Geschmack und Konsistenz länger zu erhalten, kann der Käse in ein, in Salzwasser getränktes, Geschirrtuch gewickelt werden. Dieses muss gut auswringen, den Käse einschlagen 
und alle zwei Tage gewechselt werden. Abgepackter Käse sowie Frischkäse sollte innerhalb einer Woche verzehrt werden, da diese bei geöffneter Verpackung dazu neigen auszutrocknen und ihr Aroma verlieren. Meist können sie aber auch nach einer Woche noch problemlos verzehrt werden. Durch unterschiedlichen Verfahren bei der Aufbereitung von Milch ist diese unterschiedlich haltbar. Beispiels weise ist pasteurisierte Frischmilch auch ungeöffnet im gekühlte Zustand nur 6 Tage haltbar. Ultrahoch erhitzte Milch hingegen kann ungeöffnet bei Zimmertemperatur bis zu 8 Wochen gelagert werden. %TODO Quelle Zu gut für die Tonne

 \paragraph{Teigwaren}
Frische Teigwaren aus der Kühltheke sind sehr anfällig für Keime. Nachdem öffnen halten sie sich meist 3 bis 4 Tage im obersten Fach im Kühlschrank. 

Gekochte Pasta die, nachdem kochen, übrig bleibt, kann bis zu zwei Tagen im Kühlschrank aufbewahrt werden. Voraussetzung dafür ist, dass diese bissfest gekocht wird und nachdem kochen mit kaltem Wasser übergossen wird. Anschließend zwei bis drei Stunden auskühlen lassen, damit kein Schwitzwasser entsteht was die Konsistenz sehr weich werden lässt. Danach sollte sie in einem geschlossenen Behälter im Kühlschrank aufbewahrt werden. 

Getrocknete Nudeln sollte bei Raumtemperatur, trocken und gut verschlossenen gelagert werden, damit sie nicht von Schädlingen wie Motten angegriffen werden. %TODO Quelle Zu gut für die Tonne

 \paragraph{Fleisch und Fisch}
Fleisch und Wurst zählen zu den leicht verderbliche Lebensmittel, die, nachdem Einkauf direkt in die Kühlung müssen. Bei warmen Außentemperaturen, sollte Fleisch bereits auf dem Weg nach Hause von Einkauf kühl transportiert werden z. B. in einer Kühltasche oder -box.
Rohes Fleisch bleibt am längsten frisch, wenn es auf einem Teller mit Klarsichtfolie abgedeckt im unteren Kühlschrankfach gelagert wird. Der Fleischsaft dabei aber nicht mit anderen Lebensmitteln in Berührung kommen, da sich darin Bakterien sammeln können.
Wurstaufschnitte sowie angebrochene Gläser mit Kochwurst, müssen ebenfalls in dem Kühlschrank aufbewahrt werden. Genauso wie Fleisch im untersten Fach.
Wegen der kurzen Haltbarkeit empfiehlt sich, Wurst und Fleisch in kleinen Mengen zu kaufen und schnell zu verbrauchen. %TODO Quelle Zu gut für die Tonne

Auch Fische zählen zu den leicht verderblichen Lebensmitteln. Nachdem Kauf sollte dieser direkt zubereitet werden und nicht länger als einen Tag im Kühlschrank gelagert werden. Bei der Lagerung im Kühlschrank sollte Fisch abgedeckt in einer Schale oder tiefem  Teller  auf der Glasplatte nahe der RÜckwand aubewahrt werden. Fisch kann auch problemlos eingefrohren werden, dafür sollte dieser frisch, ausgenommen, gewaschen und verpackt seien. 


 \paragraph{Obst und Gemüse}
 
 Der Großteil an Gemüse kann im Kühlschrank gelagert werden, ausnahmen sind da Auberginen, Tomaten, Kartoffeln und Kürbis. Die Lagerung von Obst ist hingegen komplexer, denn manche Sorten bewahren nur gekühlt frische und Vitamine wohingegen andere empfindlich auf Kälte reagieren. Heimisches Obst z. B. Apfel, Kirsche oder Zwetschgen können im Kühlschrank aufbewahrt werden. Exotische Früchte wie Mangos, Bananen oder Zitrusfrüchte vertragen keine Kälte. Die Ausnahme bilden Kiwi und Feigen. Auch zu beachten ist das einige Obst- und Gemüsesorten, im Reifeprozess das Gas Ethylen freisetzen. Dadurch verdirbt und altert anderes Obst und Gemüse in der Nähe schneller. Ethylen produzieren besonders Tomaten, Äpfel, Aprikosen und Pflaumen. Getrennte Aufbewahrung dieser verlängern die Haltbarkeit andere Obst- und Gemüsesorten im Haushalt. Obst und Gemüse eignet sich da drüber hinaus für verschiedene häusliche Verarbeitungsprozesse die unter "Verlängerung der Haltbarkeit" näher erläutert werden.  %TODO Quelle Zu gut für die Tonne


\subsection{Einkaufsplanung}
Lebensmittelverluste können in Privathaushalten durch eine sorgfältige Planung eingegrenzt werden. Dies beginnt vor dem Einkauf, mit Überprüfung der Vorräte im Kühlschrank und in der Vorratskammer. Dabei kann das Erstellen von Vorratslisten hilfreich sein. Auf Grundlage dessen kann ein Einkaufszettel geschrieben werden, mit dem unnötige Lebensmittelkäufe verhindert werden. Bei der Planung des Einkaufes sollten auch, gerade wenn für die ganze Woche eingekauft wird, Pläne wie Feiern oder auswärts essen eingeplant sein. Während des Einkaufs selber ist es wichtig Preise und Qualität in Ruhe zu vergleichen und bewusst die Lebensmittel auszuwählen. Bei Angeboten, gerade groß Verpackungen, sollte kritisch hinterfragt werden, ob die gekaufte Menge vor ablauf der Haltbarkeit verzehrt werden kann oder ob das Produkt überhaupt benötigt wird. %TODO Quelle(QUelle: 10Regeln) 
Gerade ein günstiger Preis verlockt oft Produkte zu kaufen die gar nicht benötigt werden.

Oft hilft es häufiger einzukaufen, aber kleine Mengen zu kaufen. So sind die Produkte frischer und werden zeitnah verbraucht. %TODO Quelle

\subsection{Verfahren zur Verlängerung der Haltbarkeit}
\subsubsection{Einkochen}
\paragraph{Definition und Prozess}
Der Prozess des Einkochens bedeutet die Lebensmittel in eine sauberen und luftdichten Einmachglas, innerhalb eines Einkochtopfs mit Wasser zu erhitzen. Die Temperatur sollte zwischen 75 und 120 Grad betragen. Die Dauer des Einkochens kann je nach Lebensmittel 10min bis zu 2 Stunden betragen. Alternativ können Obst und Gemüse auch im Backofen oder in der Mikrowelle eingekocht werden.
Während des Einkochens kommt es dazu das sich Luft und Wasserdampf im Glasbehälter ausdehnen. Die Wärme sorgt für einen Überdruck, zeitgleich werden zwischen dem Gummiring und dem Glasrand Dampf, heiße Luft und eventuell Flüssigkeit aus dem Glas herausgedrückt. Innerhalb des Einkochtopfs gelangen weder Luft noch Wasser in das Glas. Abschließend entsteht durch den Unterdruck im Glas während des Abkühlens ein Vakuum, das den Deckel luftdicht verschließt. Diese sorgt für die verlängerte Haltbarkeit.   %TODO Quelle (quelle: https://www.bmel.de/SharedDocs/Downloads/DE/Broschueren/Kompassernaehrung/Kompassernaehrung-Ausgabe-2-2018.pdf?__blob=publicationFile&v=5) 

\paragraph{Lebensmittel die sich zum Einkochen eignen}
Fast alle Lebensmittel können durch das Einkochen länger haltbar gemacht werden. Bedingung dafür ist das die Lebensmittel frisch und einwandfrei seien sollten. Neben Obst, Gemüse können sogar Fleisch, Wurst oder Kuchen eingemacht werden. Bei Obst ist zu beachten das dieses nur roh verwendet werden darf, Gemüse darf auch blanchiert sein. 

\paragraph{Beim Einkochen beachten}
Eingekochtes Gemüse und Fleisch, sollte nach mindestens 24 Stunden erneut eingekocht werden. Da Sporen eventuell erneut aufkeimen können und es nach dem Verzehr zu einer Lebensmittelvergiftung führt. Des weiteren sollten eingekochtes Gemüse und Fleisch, aus dem selben Grund vor dem Verzehr ebenfalls erhitzt werden. Ebenfalls zu beachten ist, sollte während des Einkochens eins der Gläser aufgeht, muss dieses entsorgt werden da die Lebensmittel nicht mehr im Einwandfreien Zustand sind. 


 %TODO Quelle

\subsubsection{Fermentierung}
\paragraph{Definition und Prozess}
Fermentierung ist im eigentlichen Sinne die Umwandlung von organischen Stoffen unter Einfluss von Bakterien, Pilzen, Zellkulturen oder zugesetzten Enzymen. Dies kann im Zusammenspiel mit (aerob) oder ohne Sauerstoff(anaerob) geschehen. Im Kontext mit Lebensmitteln ist Fermentierung mit Vergärung gleichgesetzt und beschrieb eine altbekannte Konservierungsmethode. %TODO Quelle (quelle : https://www.bzfe.de/inhalt/vergaeren-1351.html)

Besonders für eine Fermentierung in der heimischen Küche eignet sich reifes Gemüse. (quelle zu gut für die Tonne) Dies muss geschnitten, geraspelt oder gestampft, unter Zugabe von Salz und Wasser in ein sauberes Gefäßes gefüllt werden. (Kompass Ernährung 2/2018) Während des Prozesses der Gärung fällt der pH-Wert im Lebensmittel und verhindert das Wachstum von Mikroorganismen. %TODO Quelle (quelle : https://www.bzfe.de/inhalt/vergaeren-1351.html) 
Zeitgleichen wandeln die Milchsäurebakterien im Lebensmittel Kohlenhydrate, Proteine und Fette in Gase, Alkohol oder Säuren um. %TODO Quelle (Kompass Ernährung 2/2018)
Der Gärungsprozess muss stetig kontrolliert werden, denn es bildet sich in Form eines graues Häutchen an der Oberfläche, Kahmhefen. Diese kann den gärenden Lebensmitteln einen unangenehmen Geschmack verleihen und die Milchsäure im Gefäß entziehen. Die Kahmhefe muss bei den meisten Gemüsesorten täglich entfernt werden. Zu Beginn der Fermentierung werden die Lebensmittel zwei Tage Zimmertemperatur ausgesetzt. Dadurch setzt der eigentliche Prozess erst ein, der sich durch aufsteigende Bläschen bemerkbar macht. Anschließend muss das Vergärung Gut mindestens vierzehn Tage bis zu sechs Wochen an einem Kühlen Ort gelagert werden.



\paragraph{Haltbarkeit}
 Ein luftdichtes Glas mit fermentierten Gemüse ist für mehrere Monate haltbar. Die Haltbarkeit kann durch zusätzliches Einkochen nach der Fermentierung verlängert werden. Ein geöffnetes Glas mit fermentierten Lebensmitteln kann im Kühlschrank vier bis sechs Wochen aufbewahrt werden. %TODO Quelle (quelle : https://www.bzfe.de/inhalt/vergaeren-1351.html)
\subsubsection{Trocknen}
\paragraph{Definition und Prozess}
Wie das Fermentieren ist das Trocknen von Lebensmitteln ein traditionelles Verfahren um die Haltbarkeit von Lebensmitteln zu verlängern. (https://www.bzfe.de/inhalt/trocknen-1349.html) Bei dem Vorgang werden Lebensmittel durch Wärmeeinwirkung Wasser entzogen. Meist bleibt in den Lebensmitteln eine Feuchtigkeit von 8 bis 20 Prozent erhalten. Mikroorganismen können sich dadurch nicht mehr weiter vermehren. Was zu Folge hat, dass das Lebensmittel länger haltbar sind. Häufig wird für diesen Prozess ein Dörrautomat verwendet, Lebensmittel können aber auch im Backofen (bei leicht geöffneter Backofentür) oder an der Luft (Raumtemperatur muss ca. 30 Grad betragen) getrocknet werden.  %TODO Quelle (https://www.bzfe.de/inhalt/trocknen-1349.html)

\paragraph{Lebensmittel die sich zum Trocknen eignen}
Besonders beliebte Lebensmittel für den Prozess der Trocknung ist Obst, da sich ein intensiveres Aroma entwickelt und Vitamine sowie Ballaststoffe konzentriert vorliegen. Aber auch Gemüse und Kräuter können problemlos getrocknet werden %TODO Quelle (Kompass Ernährung 2/2018)

\paragraph{Lagerung und Haltbarkeit}
Getrocknete Lebensmittel sollten möglichst dunkel und trocken gelagert werden in Luftdicht Verschließbaren Gefäßen. Bei dieser Lagerung halten sich getrocknete Lebensmittel sechs bis zwölf Monate. %TODO Quelle (Kompass Ernährung 2/2018)

\subsubsection{Einlegen}
Durch das Einlegen von Gemüse in Essig entsteht ein saures Milieu, bei dem Mikroorganismen am Wachstum gehindert werden. Doch dies geschieht erst bei einer Konzentration von zwei bis neun Prozent. Dies ist für den menschlichen Geschmack jedoch zu sauer, daher wird meist eine Essigsäurekonzentration von einhalb bis drei Prozent verwendet. %TODO Quelle $(https://www.bzfe.de/inhalt/einlegen-1350.html)$ 
Im Kühlschrank hält sich das eingelegte Gemüse ca. ein bis zwei Wochen. Durch die Kombination mit anderen Konservierungsverfahren z. B.: Einkochen, Salzen oder Zuckern können sich die eingelegten Lebensmittel (je nach Essig-Zucker-Konzentration) mehrere Monate halten. %TODO Quelle $(https://www.bmel.de/SharedDocs/Downloads/DE/Broschueren/Kompassernaehrung/Kompassernaehrung-Ausgabe-2-2018.pdf?__blob=publicationFile&v=5)$

In Essig einlegen lassen sich besonders feste Gemüsesorten wie z. B. Rote Bete, Paprika, Gurke oder Möhren.

\paragraph{Einlegen in Öl}
Das Einlegen in Öl ist eine sehr alte Konservierungsmethode, dennoch ist diese kein sicheres Konservierungsverfahren für Privathaushalte. Von der Herstellung von Gemüse oder Kräuter in Öl wird von dem Bundesinstitut für Risikobewertung (bfR) abgeraten, da es zu einer Lebensmittel-Vergiftung kommen kann. %TODO Quelle

\paragraph{Einlegen in Alkohol}
Ähnlich wie Essig hemmt auch Alkohol das Wachstum von Mikroorganismen bzw. bei hoher Konzentration tötet Alkohol diese ab. Zusätzlich kann bei dem einlegen in Alkohol noch Zucker verwendet werden, diese reduziert den Wassergehalt in den Lebensmitteln was die Haltbarkeit verlängert. %TODO Quelle

Als Spirituosen für das Einlegen eignen sich hochprozentige Spirituosen bzw. Weinbrand, Wodka oder 54-prozentigem Rum. Der Alkoholgehalt ist wichtig, dass ein zu niedriger die Früchte gären lässt und durch einen zu hohen Gehalt sie zu fest werden. %TODO Quelle
 
Für diesen Prozessen lassen sich beinahe alle Obstarten verwenden, mit Ausnahme von Stachelbeeren, Preiselbeeren, Heidelbeeren und Weintrauben. Die Haltbarkeit für in Alkohol eingelegtes Obst beträgt ca. ein Jahr, danach dickt die Flüssigkeit ein und die Früchte werden hart. %TODO Quelle

\subsubsection{Pökeln}
\paragraph{Defintion}
Das Pökeln ermöglicht Fleisch mithilfe einer Kochsalz- oder Nitritpökelsalzlösung lange haltbar zu machen. Durch das Einreiben mit der Lösung, wird dem Fleisch das Osmose Wasser entzogen. Das beutetet das die Wassermoleküle aus den Zellen des Fleisches durch Bewegung in der Salzlake eine gleich hohe Konzentration in den Zellen erreichen wie in der Lake. Dieser Prozess entzieht den Mikroorganismen die Lebensgrundlage, dadurch werden sie am Wachstum gehemmt.

Durch das Pökeln kann das Fleisch seine Farbe sowie eine gewisse Aromatisierung erhalten. Dafür wird das Kochsalz mit Salzen der Salpetersäure vermischt. Falls die Farbe nicht erhalten muss, reicht es auch das Fleisch zu salzen.

\paragraph{Pökelverfahren}
Es können drei verschiedene Methoden zur Pökelung verwendet werden:

\begin{itemize}
  \item Trockenpökeln: Für die Qualitativ besten Ergebnisse sorgt dieses Verfahren. Das Fleisch wird traditionell per Hand mit der Salzmischung eingerieben. Anschließend wird das Fleisch in Stellagen oder Bottichen gestapelt, dabei entsteht Eigenllake.
  \item Nasspökeln: Nachdem einreiben mit der Salzmischung wird das Fleisch bei diesem Verfahren in Pökellake eingelegt.
  \item   Impfpökelung: Typisches Verfahren für Kochpökelware, durch Ader- oder Muskelinjektion gelangt der Pökelstoff in die Fleischstücke.
 
\end{itemize}


\paragraph{Bei dem Verzehr beachten}
Durch den hohen Salzgehalt des Pökelfleisches sollten Personen mit Bluthochdruck, der auf eine Nierenfunktionsstörung zurückzuführen ist, ihren Konsum dessen kontrollieren. Des Weiteren darf Pökelfleisch nicht auf den Grill. Das Nitrit kann in Kombination mit dem im Fleisch enthaltende Eiweiß bei hohen Temperaturen zu dem schädlichen Nitrosaminen werden.

%to do (quelle: https://www.bzfe.de/nachhaltiger-konsum/haltbarmachen/salzen-und-poekeln/)

\subsubsection{Einfrieren}
Um Nährstoffe, Aroma und Aussehen eines Nahrungsmittels länger zu erhalten eignet sich das Einfrieren. Durch den Wärmeentzug, auf mindestens -18 Grad, wird nämlich die Aktivität der Lebensmittel-Enzyme reduziert. Außerdem wird das Wachstum von Mikroorganismen unterbrochen. So wird die Qualität und Konsistenz des Lebensmittels erhalten. %TODO Quelle


\paragraph{Vorbereitung}
Obst und Gemüse sollten vor dem Einfrieren gewaschen und geputzt werden. Am besten ist, wenn diese vorportioniert werden, gleich gilt auch für Fleisch und Fisch. %TODO Quelle

Gempse kann vor dem Einfrieren blanchiert werden, dies spare bei der später Zubereitung Zeit. Hinzukommen bleibt dadurch die Farbe des Lebensmittels und der Vitamine C Gehalt besser erhalten. Das Gemüse sollte möglichst immer ungewürzt gefroren werden, den Salz entzieht dem Lebensmittel Wasser. Auch bei fertig gegarten Speisen sollte das Würzen eher gering ausfallen, wenn diese eingefroren werden soll, den Gewürze wie Pfeffer, Paprika, Muskat verlieren ihren Geschmack wohingegen Basilikum Thymian oder Dill intensiver schmecken können. %TODO Quelle

\paragraph{Verpackung}
Beim Einfrieren sollte auch auf eine geeignete Verpackung geachtet werden. Diese verhinderte Qualitätsverlust wie zum Beispiel Geruchs- oder Geschmacksveränderungen und Gefrierbrand.
%TODO Quelle

Folgende Behälter eignen sich:
\begin{itemize}
  \item Hitze- und kältebeständiges Glas (Schneeflocken Symbol auf dem Behälter weist auf die Eignung hin)

  \item Dosen und Schalen aus Kunststoff (Ebenfalls Schneeflocken Symbol beachten)
  \item spezielle gefriergeeignete Folienbeutel aus Polyethylen und Gefrierkochbeutel(sind meist nur einmal verwendbar). Bei der Verwendung eines Gefrierbeutels ist zu beachten die Beutelenden mit Clips oder Klemmen fest verschlossen werden sollten.
  \item   Alufolie eignet sich ebenfalls, wenn keine andere Möglichkeit durch die Form des Gefrierguts vorhanden ist. Dennoch sollte diese nur in Ausnahme verwendet werden, da diese energieaufwendig hergestellt wird aber auch durch Salz und Säure aus dem Lebensmittel Löcher in die Alufolie geraten und sich somit Aluminiumbestandteile auf dem Lebensmittel 
\end{itemize} %TODO Quelle


%TODO Quelle


\paragraph{Lagerdauer}
Wie bereits aufgeführt gehen die Abbauprozesse von tiefgefrorenen Lebensmittel nur langsamer weiter. Daher ist es wichtig die Lagerdauer zu berücksichtigen. Hilfreich ist es auch ein Bestandsverzeichnis zu erstellen oder die Nahrung von dem Einfrieren mit einem Datum zu kennzeichnen. 
Ein grobe Übersicht zu der Lagedauer:
\begin{itemize}
  \item fertige Speisen 3 Monate
  \item Fleisch: 3 - 12 Monate
  \item Gemüse: 6 - 12 Monate
  \item Obst: 8 - 12 Monate
\end{itemize}

\paragraph{Auftauen}
Die aufgetauten Lebensmittel sollten möglichst schnell weiterverarbeitet und verbraucht werden, da bei steigender Temperaturen sich die Mikroorganismen wieder vermehren.

Zuvor Blanchiertes gefrorenes Gemüse kann direkt im Topf oder in der Mikrowelle zubereitet werden ohne aufgetaut werden zu müssen. Obst solle bei Raumtemperaturen oder in der Mikrowelle aufgetaut werden. Kartoffelprodukte und Backwaren sollten im Backofen aufgetaut werden.

Bei Fleisch und Fisch muss beachtet werden das sich Mikroorganismen in der Auftauflüssigkeit befinden können. Diese sollte abgedeckt im Kühlschrank oder in einer Mikrowelle aufgetaut werden. Dabei sollte der Kontakt mit anderen Lebensmittel verhindert werden. Die Flüssigkeit sollte direkt nach dem Auftauen entsorgt werden. Rohe Meeresfrüchte können hingegen unaufgetaut gegart werden.

Wichtig ist zu beachten, dass die angetauten bzw. aufgetauten Lebensmittel nur bei einwandfreiem Zustand wieder eingefrohren werden können. Denn bei einem zweiter Einfrierprozess kann zu geschmacklichen und qualitativen Einbußen kommen.\paragraph{Nicht einzufriedende Lebensmittel}
Nicht alle Lebensmittel dürfen eingefrohren werden, bei folgenden Lebensmitteln ist Vorsicht geboten: 
\begin{itemize}
  \item Wasserreiche Lebensmittel verlieren ihre typische Konsistenz durch das Einfrieren und werden beim Auftauen z. B. matschig Dadrunter fallen Salate, bestimme Gemüsesorten (Rohe Tomaten, Paprika, Gurke), Obst (Ganze rohe Äpfel, Birnen, Weintrauben, Melonen)
  \item Milchprodukte neigen dazu auszuflocken und eine grießige Konsistenz zu bekommen (Dickmilch, Saure Sahne, Jogurt, Crème fraîche)

  \item Rohe Eier platzen in der Schale und gekochte werden glasig

  \item Pudding oder Gelantienhaltige Speisen können wässrig werden.
\end{itemize}

 

\subsection{Kompostierung}
Kompostierung beschreibt ein natürliches Recyclingverfahren. Dabei verrotten Garten- und Küchenabfälle im eigenen Garten auf einem Kompostplatz zu Humus. Neben der Weiterverarbeitung von Lebensmittelresten, sparen kompostierten Abfälle Lärm- und Schadstoffemissionen, dadurch das sie nicht abtransportiert und verarbeitet werden müssen. Zu beachten ist das Kompostierung im eigenen Garten nur Sinn macht, wenn der Bedarf vorhanden ist.
Wichtig bei der Kompostierung ist, den richtigen Platz zu wählen: (halb)schattiger auf offenem Boden. Als Behälter kann zwischen offenen Systemen (Miete, Draht- oder Latten-Komposter) und geschlossenen Schnell- oder Thermokompostern gewählt werden. Alle Aufbewahrungen bieten ihre Vor- und Nachteile, grundsätzlich ist nur zu beachten das der Kompost feucht (nicht nass) und durchgelüftet aufbewahrt wird. %TODO Quelle $(quelle: https://www.umweltbundesamt.de/umwelttipps-fuer-den-alltag/garten-freizeit/kompost-eigenkompostierung#gewusst-wie) $
Kompostierung ist eine gute Möglichkeit Lebensmittelreste sinnvoll weiter zu verwerten, dennoch muss dafür ein eigener Garten gegeben sein, um diese herzustellen und zu verwenden.

\subsection{Wertschätzung von Lebensmitteln}
Um Lebensmittelverschwendung zu verringern, muss auch das Bewusstsein für den Wert von Lebensmitteln geschaffen werden. Denn nach Kranert, wird "die Entfremdung gegenüber Lebensmitteln weiter zunehmen" wird (...) S.218. \subsubsection{Ernährungsbildung}
Bereits im frühen Alter sollte Wertschätzung für Lebensmittel gelernt werden. Das Bundeszentrum für Ernährung setzt sich dafür ein und bietet Unterrichtsmaterialien im Modul Ernährungsbildung für Kitas bis hin zur Sekundarstufe eins an. %TODO Quelle(https://www.bzfe.de/inhalt/sinnesbildung-1581.html) (https://www.bzfe.de/inhalt/wertschaetzung-von-lebensmitteln-2361.html)
Besonderer Fokus liegt bei auf dem nachhaltigen Einkaufen. Saisonal, regional und Bio sind dabei die Schlüsselbegriffe. Veranschaulicht sollen diese durch Unterrichtsbesuchen zum Bauernhof, Bioladen, Wochenmarkt oder Gemüsehändler empfiehlt das Bzfe. Das Kennenlernen und wertschätzen von unperfektem Obst und Gemüse, das genauso schmeckt wie aus dem Supermarkt, steht dabei im Vordergrund. %TODO Quelle https://www.bzfe.de/inhalt/wertschaetzung-von-lebensmitteln-2361.html
Ein weiteres Projekt im Auftrag des NRW-Verbraucherschutzministeriums, durchgeführt der Universität Paderborn und Verbraucherzentrale NRW trägt zur Ernährungsbildung bei. Ein Werkzeugkoffer für Lehrer*innen wurde entwickelt, der rund 19 Themen-Bausteine beinhaltet und Wertschätzung von Lebensmittel in den Unterricht bringt. %TODO (quelle:https://www.umwelt.nrw.de/verbraucherschutz/konsum-und-wertschaetzung-von-lebensmitteln) 
\subsubsection{Informations Kampagne}
Um die Öffentlichkeit für das Thema Lebensmittelverschwendung zu sensibilisieren, können Informationskampagne sinnvoll sein. Es gibt wie bereits im Kapitel Politik aufgeführt einige Initiativen und auch Kampagnen seitens des Umweltministeriums. Dennoch sind diese Informationen nicht in der breiten Masse bekannt. Erkenntnisse aus Studien müssen zum Allgemeingut werden. Gerade, dass Hinweise auf finanzielle Vorteile können ein ausschlagen sein sich sorgsamer mit Lebensmitteln auseinander zu setzten. Beispielsweise kann eine vierköpfige Familie bis zu 1.200 Euro im Jahr einsparen, wenn sie verlustarmer kocht. %TODO (wwf)


\subsection{Innovation für einen nachhaltigen Lebensmittelumgang}
Für Konsumenten sind in den letzten Jahren  verschiedene Innovationen entstanden, mit denen sie Aktiv Lebensmittelverschwendung entgegenwirken können. 

\subsubsection{Too Good To Go}
Mit “Too Good to Go2 ist es den Konsumenten möglich übrig gebliebene Lebensmittel aus Resturants, Bäckerrein und Supermärkten zu retten. Die Nutzer der App können zu einem wesentlich geringeren Preis die Waren zu einer vorgegeben Zeitpunkt in dem Ladenlokal abholen. 2019 wurden die Erfinder der App mit dem “Zu gut für die Tonne! - Bundespreis 2019” in der Kategorie Digitalisierung geehrt. (Quelle https://www.bundesregierung.de/breg-de/themen/digitalisierung/apps-lebensmittelverschwendung-1653238) 
\subsubsection{Beste Reste}
Die kostenlose App gehört zu der Kampagne “Zu gut für die Tonne!” der Bundesregierung und bietet mehr als 700 Kochrezepte für Lebensmittelreste. Innerhalb der App lassen sich per Suchfunktion bis zu drei Zutaten miteinander kombinieren um Rezepte zu erhalten.
\subsubsection{Lebensmittel- und Kochboxen}
Etepetete liefert regionales Obst und Gemüse, das während der Ernte aufgrund von Qualitätsansprüche des Handels aussortiert wurde, direkt an den Kunden. Dabei legt das Unternehmen Wert auf eine umweltfreundliche Verpackung, effizienten Lieferkette und klimaneutrale Prozesse.

Das Berliner Start-up SIRPLUS legt seinen Fokus ebenfalls auf überschüssige Ware, die nicht der Norm entspreche oder vor dem Ablauf des Mindesthaltbarkeitsdatums stehe. Dabei handele es sich um verzehrfertige Produkte aus der Lebensmittelindustrie. Über den Online-Shop besteht die Möglichkeit "Retterboxen" zu bestellen die in regelmäßigen Intervallen geliefert werden und verschiedene Produkte erhalten.

Kochboxen Anbieter wie Hello Fresh oder Marley Spoon werben vor allem mit der Zeitersparnis ihres Services. Dadurch das die Anzahl der Portionen bei dem Anbieter festgelegt werden, erhält der Konsument nur die Menge für das Rezept und dementsprechend keine spezial Zutaten in großer Menge kaufen muss, die nicht aufgebraucht werden können. Dennoch ist zu beachten, dass durch die einzelnen Verpackungen von Soßen und Co. auch eine große Menge Verpackungsmüll anfällt.


\begin{itemize}
  \item {Phase 1}: Sicherer Gesprächsrahmen
  \item {Phase 2}: Sichtweisen hören und verstehen, Themen sammeln
  \item {Phase 3}: Gemeinsame Erhellung der Konflikthintergründe
  \item {Phase 4}: Alternativen schaffen
  \item {Phase 5}: Lösungsvorschläge prüfen und machbares entwickeln
  \item {Phase 6}: Vereinbarungen treffen
\end{itemize}




%\newpage
\section{Praxis}

\subsection{Segmentierungskonzept}
Im vorangegangenen Theorieteil wurde die Segmentierung und ihre Kriterien definiert. Anhand dessen soll nun ein Segmentierungskonzept für die neu entwickelte Foodwatch App entwickelt werden. Ziel ist es eine Zielgruppe bzw. Segmente zu erschließen.

\subsection{Relevanten Markt abgrenzen}
Um die Segmente zu identifizieren, muss zu Beginn der relevante Markt festgelegt werden. Die sachlich angebotene Leistung der App ist das Archivieren der Lebensmittelbestände im Haushalt, eine Reminder Funktion vor Verfall der Lebensmittel sowie passende Rezepte zu den sich im Haus befindlichen Lebensmittel. Da die Konsumenten regelmäßig einkaufen, ist der Markt nicht zeitlich auf eine Saison begrenzt. Eine räumlich Abgrenzung erfolgt dadurch, dass die App zu Beginn nur in Deutschland verfügbar sein wird.

\subsection{Bedarf und Anforderung der Zielgruppe}
Die Leistung der App wird von Menschen benötigt, die:

\begin{itemize}
\item dazu neigen Lebensmittel im Kühlschrank oder Vorratsschrank zu vergessen.
\item bisher einen Mangel an Kreativität beim Kochen haben und dadurch oft dasselbe kochen.
\item eine Übersicht ihrer zu Hause aufbewahrten Lebensmittel benötigen.
\item Lebensmittelverschwendung im eigene Haushalt verringern wollen.
\end{itemize}

Die Anforderung an die Zielgruppe sind:
\begin{itemize}
\item technische Affinität bzw. regelmäßige Smartphone Nutzung
\item einen Sinn für Ordnung und Übersicht zu haben
\item Interesse an dem Thema Lebensmittelverwendung bzw. allgemeines Interesse an Lebensmitteln und Kochen \end{itemize}

\subsection{Bewertung der verwendeten Segmentierungskriterien}
Der Autor verwendet als Kriterium die Sinus Milieus der psychografischen Segmentierungskriterien zum Ermitteln einer Zielgruppe.

\begin{itemize}

\item \textbf{Relevanz im Kaufverhalten:} Da die Sinus Milieus zu den psychografischen Segmentierungskriterien zählen, wird sich mit nicht beobachtbaren Faktoren des Kaufverhalten befasst.  Der kausaler Zusammenhang zum Kaufverhalten wird durch Interpersonale Faktoren erzeugt. 
\item \textbf{Messbarkeit:} Die Messbarkeit erfolgt zum einen durch die Downloads der App, zum anderen durch den Anteil den die gewählten Milieus an der Bevölkerung haben. 
\item \textbf{Zeitliche Stabilität:} Die Sinus Milieus existieren bereits seit 40 Jahren und sind dadurch zeitlich Stabil. \footcite[Vgl.][ ]{website:sinus-institut}
\item \textbf{Umsetzbarkeit:} Für das Erreichen der Milieus bzw. der Segmente gibt es reichlich Fachliteratur und auch das Sinus Institut selbst bietet als Leistung die Erstellung eines individuellen Marketingkonzeptes.
\item \textbf{Wirtschaftlichkeit:} Dadurch dass die Sinus Milieus reale Personengruppen darstellen, ist die Wirtschaftlichkeit gewährleistet.

\end{itemize}

\subsection{Zielgruppen nach Sinus Milieus}
Als mögliche Nutzer der App fokussiert der Autor das Milieu der Performer, das sozioökoligsche Milieu sowie das adaptiv-pragmatische Milieu.

\subsubsection{Milieu der Performer}
Die Performer sind die Leistungselite. \footcite[Vgl.][ ]{website:sinus-institut}
Sie zeichnen sich durch Globalökonomisches Denken und ihr nach Effizienz orientiertes Handeln aus. \footcite[Vgl.][ ]{website:sinus-institut}
Die Performer sehen sich selbst als Konsum- und Stilavantgarde, mit einer hohen Affinität zur aktuellsten Technik. \footcite[Vgl.][ ]{website:sinus-institut}
Bei der Nutzung der digitalen Medien gehen sie effizient vor und kombinieren selbstverständlich offline und online Angebote. \footcite[Vgl.][ ]{website:sinus-institut}
Als Zielgruppe kommen sie in Frage, da sie auch im privaten Bereich immer wieder nach Optimierung streben. Die umfangreichen Funktionen der App sind vorteilhaft für die Personen aus diesem Milieu.

\subsubsection{Sozialökologisches Milieu}
Als engagiert gesellschaftskritisches Milieu mit normativen Vorstellungen vom richtigen Leben haben Menschen dieses Milieus ein ausgeprägtes, ökologisches und soziales Gewissen.\footcite[Vgl.][ ]{website:sinus-institut}
Relevant als Zielgruppe sind sie vor allem durch ihr ökologisches Gewissen. Personen des sozialökologischen Milieus ist ihre Umwelt wichtig und sie wollen nachhaltig und sinnvoll mit ihren Ressourcen umgehen. 
Dieses Milieu nutzt eher selektiv online Medien, \footcite[Vgl.][ ]{website:sinus-institut} daher ist es bei der Ansprache wichtig, die ihnen vertrauten online Medien zu verwenden.

\subsubsection{Adaptiv Pragmatisches Milieu}
Die moderne Junge Mitte nimmt den Platz des Mainstream innerhalb der jungen Freizeitkultur ein. \footcite[Vgl.][S. 37]{Barth2017} 
Die adaptiv Pragmatischen zeichnen sich durch Nützlichkeitsdenken und Pragmatismus aus, gerade im Hinblick auf ihre Zielerfüllung. \footcite[Vgl.][S. 37]{Barth2017} 
Spaß, Komfort und Unterhaltung sind ihnen trotzdem wichtig und dienen als Ausgleich zu ihres Leistungsbereitschaft im Berufsleben. \footcite[Vgl.][S. 37]{Barth2017}
Ebenfalls wie die Performer, ist das adaptiv pragmatische Milieu sicher im Umgang mit digitalen Medien. \footcite[Vgl.][ ]{website:sinus-institut}
Das Nützlichkeit-Orientierte Denken dieses Milieus, macht sie zu einer relevanten Zielgruppe für die fikitive Foodwatch App mit ihrem umfangreichen Repertoire an Funktionalitäten.\footcite[Vgl.][S. 36]{Barth2017} 

Die Milieus sind untereinander heterogen, aber ergeben zusammen eine homogene Gesamtgruppe. Dies resultiert daraus, da sie auch auf der Kartoffelgrafik nah beieinander liegen und Schnittpunkte aufweisen.

\subsection{Segmente}

Aus den Sinus Milieus ergeben sich die folgenden drei Segmente:
\begin{itemize}

\item  Effizient orientierte Nutzer die Optimierung anstreben,
\item Nutzer mit einem ökologischen Denken, die vor allem den Aspekt der Lebensmittelverschwendung im Fokus haben
\item  Pragmatische Nutzer, denen der Ergebnisfaktor der App wichtig ist.
\end{itemize}

%

\newpage
\section{Fazit}

Im ersten Kapitel wurden die Ursachen innerhalb der Wertschöpfungskette ausführlich aufgeführt. Dabei fällt besonders auf das zum Ende der Kette hin der Verlust immer weiter ansteigt. Dennoch muss der Nahrungsmittelverlust auf alle Stufen in gemindert werden. Dafür benötige es langfristige Konzepte, regelmäßige Maßnahmen oder finanzielle Anreize innerhalb der Landwirtschaft, Lebensmittelproduktion und im Handel. Bei den Groß- und Einzelverbrauchern sind es vor allem bewusstseinsbildenden Maßnahmen sowie Beobachtung und reflektieren des Konsument bzw. des eigenen Verhaltens in Bezug auf Nahrungsmittel.

Die Politik konzentriere sich bei ihren Maßnahmen sehr stark auf Privathaushalte. Dies machen zwar als einzelner Sektor einen großen Bereich aus, dennoch kann das ehrgeizige Ziel die Lebensmittelabfälle bis 2030 zu halbieren nicht alleine dadurch das die Reduzierung von Abfällen in Privathaushalten gestemmt werden. Dadrüber hinaus existieren keine festgelegten Zielvorgaben oder Handlungsempfehlungen für alle Beteiligten der Wertschöpfungskette. Es ist erforderlich das die Politik Strukturen und Rahmenbedingungen schafft, die nachhaltigen Konsum möglich machen und fördern.

Innerhalb des Praxisteils lag der Fokus auf die Handlungsempfehlungen zur Eindämmung von Lebensmittelverschwendung für die Privathaushalte. Dabei ist besonders hervorgetreten das kaum Programme gibt die sich für die Wertschätzung von Lebensmittel einsetzt bzw. eine entsprechende Ernährungsbildung innerhalb der Schulen fehlt. Die Beziehung zu unser Lebensmittel sowie ihrer Herkunft und welche Arbeit im Anbau bzw. Aufzucht steckt, ist essenziell für den Aufbau der Wertschätzung.

%-----------------------------------
% Literaturverzeichnis
%-----------------------------------
%\newpage
%%\addcontentsline{toc}{section}{Literatur}
%
%\pagenumbering{Roman} %Zähler wieder römisch ausgeben
%\setcounter{page}{7}  %Zähler manuell hochsetzen

%\printbibliography[heading=bibintoc,title={Literaturverzeichnis}]
%\printbibheading[title={Literaturverzeichnis}]
%\printbibliography[type=article,heading=subbibliography,title={Artikel}]
%\printbibliography[type=inproceedings,heading=subbibliography,title={Inproceedings}]
%\printbibliography[type=book,heading=subbibliography,title={Bücher}]
%\printbibliography[type=online,heading=subbibliography,title={Webseiten}]

% Alternative Darstellung:
% Literaturverzeichnis nach Typ (@book, @arcticle ...) sortiert.
% Dazu die Zeile (\printbibliography) auskommentieren und folgenden code verwenden:

%\printbibheading
%\printbibliography[type=article,heading=subbibliography,title={Artikel}]
%\printbibliography[type=book,heading=subbibliography,title={Bücher}]
%\printbibliography[type=online,heading=subbibliography,title={Webseiten}]

\newpage
\pagenumbering{gobble} % Keine Seitenzahlen mehr

%-----------------------------------
% Ehrenwörtliche Erklärung
%-----------------------------------
\section*{Eidestattliche Erklärung}
Ich erkläre hiermit, dass die vorliegende Arbeit selbstständig und ohne Benutzung anderer als der angegebenen Hilfsmittel angefertigt habe; die aus fremden Quellen(einschließlich elektronischer Quellen aus dem Internet) direkt oder indirekt übernommenen Gedanken sind ausnahmslos als solche kenntlich gemacht. Die Arbeit wurde bisher weder im Inland noch im Ausland in gleicher oder ähnlicher Form einer anderen Prüfungsstelle vorgelegt und auch noch nicht physisch oder elektronisch veröffentlicht.

\par\medskip
\par\medskip
Recklinghausen, 24.07.2020 \hspace{1.7cm}
 \\
\_\_\_\_\_\_\_\_\_\_\_\_\_\_\_\_\_\_\_\_\_\_\_\_ \hspace{1.5cm}
\_\_\_\_\_\_\_\_\_\_\_\_\_\_\_\_\_\_\_\_\_\_\_\_ \\

(Ort, Datum)\hspace{4.5cm}
(Eigenhändige Unterschrift)


\newpage\null\thispagestyle{empty}\newpage

\end{document}
