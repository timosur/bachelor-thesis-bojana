\section{Einleitung}
Jährlich werden weltweit rund 1,3 Milliarden Tonnen Lebensmittel entlang der Wertschöpfungskette entsorgt. Dies entspricht ein Drittel aller weltweit produzierten Lebensmittel. $ (Quelle: FAO: http://www.fao.org/fileadmin/user_upload/newsroom/docs/FAO%20ruft%20dazu%20auf%20weniger%20Lebensmittel%20zu%20verschwenden.pdf) $
$
Allein in Deutschland fallen 18 Mio. t. Lebensmittelverluste jährlich an.(quelle WWF) Doch was ist die Ursache der Lebensmittelverschwendung in Deutschland? Dokumentarfilmer Valetin Thurn, bekannt für seinen Film „Taste the Waste“, äußerte sich in einem Interview folgendermaßen: „Es ist ein System, an dem wir alle unseren Anteil haben, in dem wir im Supermarkt eben nur Produkte kaufen, die besonders schön aussehen. Das führt zu kosmetischen Standards und die sorgen für Müll nicht nur im Supermarkt, sondern schon vorher in der Landwirtschaft, wo danach bereits aussortiert wird.“ $ (Quelle: http://www.planet-interview.de/interviews/valentin-thurn/35464/) $

Das Thema Lebensmittelverschwendung findet immer wieder mediale Beachtung, nicht nur durch Thurns Film. Auch andere Medien greifen das Thema auf: Presseberichte, Zeitungsbeiträge, Filme, Fernsehdokumentationen und Radio- und Podcastbeiträge.


Ziel der Arbeit ist es, die Ursachen der Lebensmittelverluste innerhalb Deutschlands auf allen Stufen aufzuzeigen. Des Weiteren sollen auch die Umwelteffekte die aus dem Verlust resultieren, dargelegt werden. Der Einfluss der Politik bzw. ihr Engagement gegen Lebensmittelverschwendung verdient in diesem Zusammenhang eine nähere Betrachtung.

Innerhalb des Praxisteils wird der Fokus aus die Lebensmittelverschwendung innerhalb Privathaushalten gelegt. Da in diesem Segment die vermeidbaren Verluste besonders hoch sind und der Konsument durch sein Kaufverhalten auch die anderen Bereiche der Wertschöpfungsketten beeinflusst. 