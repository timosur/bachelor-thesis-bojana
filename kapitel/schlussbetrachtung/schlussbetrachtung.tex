\newpage
\section{Fazit}

Im ersten Kapitel wurden die Ursachen innerhalb der Wertschöpfungskette ausführlich aufgeführt. Dabei fällt besonders auf das zum Ende der Kette hin der Verlust immer weiter ansteigt. Dennoch muss der Nahrungsmittelverlust auf alle Stufen in gemindert werden. Dafür benötige es langfristige Konzepte, regelmäßige Maßnahmen oder finanzielle Anreize innerhalb der Landwirtschaft, Lebensmittelproduktion und im Handel. Bei den Groß- und Einzelverbrauchern sind es vor allem bewusstseinsbildenden Maßnahmen sowie Beobachtung und reflektieren des Konsument bzw. des eigenen Verhaltens in Bezug auf Nahrungsmittel.

Die Politik konzentriere sich bei ihren Maßnahmen sehr stark auf Privathaushalte. Dies machen zwar als einzelner Sektor einen großen Bereich aus, dennoch kann das ehrgeizige Ziel die Lebensmittelabfälle bis 2030 zu halbieren nicht alleine dadurch das die Reduzierung von Abfällen in Privathaushalten gestemmt werden. Dadrüber hinaus existieren keine festgelegten Zielvorgaben oder Handlungsempfehlungen für alle Beteiligten der Wertschöpfungskette. Es ist erforderlich das die Politik Strukturen und Rahmenbedingungen schafft, die nachhaltigen Konsum möglich machen und fördern.

Innerhalb des Praxisteils lag der Fokus auf die Handlungsempfehlungen zur Eindämmung von Lebensmittelverschwendung für die Privathaushalte. Dabei ist besonders hervorgetreten das kaum Programme gibt die sich für die Wertschätzung von Lebensmittel einsetzt bzw. eine entsprechende Ernährungsbildung innerhalb der Schulen fehlt. Die Beziehung zu unser Lebensmittel sowie ihrer Herkunft und welche Arbeit im Anbau bzw. Aufzucht steckt, ist essenziell für den Aufbau der Wertschätzung.