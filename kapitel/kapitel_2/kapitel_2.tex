\newpage
\section{Praxis}

\subsection{Segmentierungskonzept}
Im vorangegangenen Theorieteil wurde die Segmentierung und ihre Kriterien definiert. Anhand dessen soll nun ein Segmentierungskonzept für die neu entwickelte Foodwatch App entwickelt werden. Ziel ist es eine Zielgruppe bzw. Segmente zu erschließen.

\subsection{Relevanten Markt abgrenzen}
Um die Segmente zu identifizieren, muss zu Beginn der relevante Markt festgelegt werden. Die sachlich angebotene Leistung der App ist das Archivieren der Lebensmittelbestände im Haushalt, eine Reminder Funktion vor Verfall der Lebensmittel sowie passende Rezepte zu den sich im Haus befindlichen Lebensmittel. Da die Konsumenten regelmäßig einkaufen, ist der Markt nicht zeitlich auf eine Saison begrenzt. Eine räumlich Abgrenzung erfolgt dadurch, dass die App zu Beginn nur in Deutschland verfügbar sein wird.

\subsection{Bedarf und Anforderung der Zielgruppe}
Die Leistung der App wird von Menschen benötigt, die:

\begin{itemize}
\item dazu neigen Lebensmittel im Kühlschrank oder Vorratsschrank zu vergessen.
\item bisher einen Mangel an Kreativität beim Kochen haben und dadurch oft dasselbe kochen.
\item eine Übersicht ihrer zu Hause aufbewahrten Lebensmittel benötigen.
\item Lebensmittelverschwendung im eigene Haushalt verringern wollen.
\end{itemize}

Die Anforderung an die Zielgruppe sind:
\begin{itemize}
\item technische Affinität bzw. regelmäßige Smartphone Nutzung
\item einen Sinn für Ordnung und Übersicht zu haben
\item Interesse an dem Thema Lebensmittelverwendung bzw. allgemeines Interesse an Lebensmitteln und Kochen \end{itemize}

\subsection{Bewertung der verwendeten Segmentierungskriterien}
Der Autor verwendet als Kriterium die Sinus Milieus der psychografischen Segmentierungskriterien zum Ermitteln einer Zielgruppe.

\begin{itemize}

\item \textbf{Relevanz im Kaufverhalten:} Da die Sinus Milieus zu den psychografischen Segmentierungskriterien zählen, wird sich mit nicht beobachtbaren Faktoren des Kaufverhalten befasst.  Der kausaler Zusammenhang zum Kaufverhalten wird durch Interpersonale Faktoren erzeugt. 
\item \textbf{Messbarkeit:} Die Messbarkeit erfolgt zum einen durch die Downloads der App, zum anderen durch den Anteil den die gewählten Milieus an der Bevölkerung haben. 
\item \textbf{Zeitliche Stabilität:} Die Sinus Milieus existieren bereits seit 40 Jahren und sind dadurch zeitlich Stabil. \footcite[Vgl.][ ]{website:sinus-institut}
\item \textbf{Umsetzbarkeit:} Für das Erreichen der Milieus bzw. der Segmente gibt es reichlich Fachliteratur und auch das Sinus Institut selbst bietet als Leistung die Erstellung eines individuellen Marketingkonzeptes.
\item \textbf{Wirtschaftlichkeit:} Dadurch dass die Sinus Milieus reale Personengruppen darstellen, ist die Wirtschaftlichkeit gewährleistet.

\end{itemize}

\subsection{Zielgruppen nach Sinus Milieus}
Als mögliche Nutzer der App fokussiert der Autor das Milieu der Performer, das sozioökoligsche Milieu sowie das adaptiv-pragmatische Milieu.

\subsubsection{Milieu der Performer}
Die Performer sind die Leistungselite. \footcite[Vgl.][ ]{website:sinus-institut}
Sie zeichnen sich durch Globalökonomisches Denken und ihr nach Effizienz orientiertes Handeln aus. \footcite[Vgl.][ ]{website:sinus-institut}
Die Performer sehen sich selbst als Konsum- und Stilavantgarde, mit einer hohen Affinität zur aktuellsten Technik. \footcite[Vgl.][ ]{website:sinus-institut}
Bei der Nutzung der digitalen Medien gehen sie effizient vor und kombinieren selbstverständlich offline und online Angebote. \footcite[Vgl.][ ]{website:sinus-institut}
Als Zielgruppe kommen sie in Frage, da sie auch im privaten Bereich immer wieder nach Optimierung streben. Die umfangreichen Funktionen der App sind vorteilhaft für die Personen aus diesem Milieu.

\subsubsection{Sozialökologisches Milieu}
Als engagiert gesellschaftskritisches Milieu mit normativen Vorstellungen vom richtigen Leben haben Menschen dieses Milieus ein ausgeprägtes, ökologisches und soziales Gewissen.\footcite[Vgl.][ ]{website:sinus-institut}
Relevant als Zielgruppe sind sie vor allem durch ihr ökologisches Gewissen. Personen des sozialökologischen Milieus ist ihre Umwelt wichtig und sie wollen nachhaltig und sinnvoll mit ihren Ressourcen umgehen. 
Dieses Milieu nutzt eher selektiv online Medien, \footcite[Vgl.][ ]{website:sinus-institut} daher ist es bei der Ansprache wichtig, die ihnen vertrauten online Medien zu verwenden.

\subsubsection{Adaptiv Pragmatisches Milieu}
Die moderne Junge Mitte nimmt den Platz des Mainstream innerhalb der jungen Freizeitkultur ein. \footcite[Vgl.][S. 37]{Barth2017} 
Die adaptiv Pragmatischen zeichnen sich durch Nützlichkeitsdenken und Pragmatismus aus, gerade im Hinblick auf ihre Zielerfüllung. \footcite[Vgl.][S. 37]{Barth2017} 
Spaß, Komfort und Unterhaltung sind ihnen trotzdem wichtig und dienen als Ausgleich zu ihres Leistungsbereitschaft im Berufsleben. \footcite[Vgl.][S. 37]{Barth2017}
Ebenfalls wie die Performer, ist das adaptiv pragmatische Milieu sicher im Umgang mit digitalen Medien. \footcite[Vgl.][ ]{website:sinus-institut}
Das Nützlichkeit-Orientierte Denken dieses Milieus, macht sie zu einer relevanten Zielgruppe für die fikitive Foodwatch App mit ihrem umfangreichen Repertoire an Funktionalitäten.\footcite[Vgl.][S. 36]{Barth2017} 

Die Milieus sind untereinander heterogen, aber ergeben zusammen eine homogene Gesamtgruppe. Dies resultiert daraus, da sie auch auf der Kartoffelgrafik nah beieinander liegen und Schnittpunkte aufweisen.

\subsection{Segmente}

Aus den Sinus Milieus ergeben sich die folgenden drei Segmente:
\begin{itemize}

\item  Effizient orientierte Nutzer die Optimierung anstreben,
\item Nutzer mit einem ökologischen Denken, die vor allem den Aspekt der Lebensmittelverschwendung im Fokus haben
\item  Pragmatische Nutzer, denen der Ergebnisfaktor der App wichtig ist.
\end{itemize}
